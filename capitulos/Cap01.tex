%\chapterimage{sisal\_pb.jpg} % Chapter heading image
\chapter{O Povoamento do Sertão}

%\section*{}
Descoberto o Brazil em 1500 pelo almirante Pedro Alvares Cabral, que delle tomou posse
em nome da Corda Portugueza, não pensou esta senão em utilisar-se das suas muitas riquesas naturaes de mais facil disposição, sem cuidar de occupal-o, colonisal-o e conduzil-o ao banquete da civilisacão.

Somente trinta e quatro annos depois, em 1534, é que, aguçada a ambição dos demais
povos europêos, e a della mesmo, resolveu dividil-o em capitanias de cincoenta leguas de littoral, distribuindo-as por subditos capazes de firmar nellas o seu dominio. A da Bahia, do Tio Jequiriçá ao S. Francisco, coube a Francisco Pereira Coitinho, que se deu pressa em aproveital-a, fundando o seu primeiro estabelecimento, no logar Victoria, encostado á Graça, onde já encontrou o seu patrício Diogo Alvares Correia, o caramurú, que ahi naufragara  em 1510, e  tendo sido bem acolhido pelos naturaes do paiz, recebera por mulher, segundo  os usos  destes a Catharina Paraguassú.

Morto Pereira, que pouco ou nada pudera fazer pelo desenvolvimento da sua capitania, e paralisada senão abandonada, por muito tempo a obra de colonisação, resolveu a Coroa Portugueza avocar as capitanias, entregando-as em 1548 a Thomé de Souza, a quem nomeou Governador Geral do Brazil com séde na Bahia.

Esse foi o passo mais  seguro, mais efficaz e mais decisivo, para a definitiva conquista e  consequente povoamento da Bahia e seus sertões.  Arrojados exploradores já haviam se internado bastante e chegado ás margens do alto S.Francisco, espalhando noticias phantasticas sobre as suas riquezas em prata ouro e pedras preciosas, que, si outro effeito não tiveram, o de restimular ao menos a gente portugueza não se lhes pode negar. A nomeação de Thomé de Souza foi de certo consequencia de uma dessas phantasias, a de  Guilhen\footnote{Filipe Guilhem, espanhol que era padre/boticário/mensageiro e muito serviu à coroa Portuguesa.}, %Felipe Gulhem 
que para Portugal escrevia a miúde, mandando contar  cousas do arco da velha, iriadas, como este, eram ellas.

E por o ter sido é que veio completamente apparelhado para uma obra completa, duradoura, sem desfallecimentos. Não podia ter esquecido, como não esqueceu, um factor na epoca  de grande monta, o frade, sobretudo o jesuita, muito mais inteligente, illustrado e experimentado e por isso mesmo mais  pratico e efficiente do que outro qualquer.

Mal fundada a Villa, no mesmissimo logar  onde  está hoje  a cidade, que tem se estendido bastante, um pouco distante da  primitiva agora por ella absorvida, que, por isso mesmo, ficou chamando-se villa velha, ou arraial do Pereira, actualmente Victória, começou a obra de cathechese e colonisação. Por um lado os frades, e  por outro  destemidos exploradores, que de nada se arreceiavam, puseram-se em demanda do sertão, que palmilhavam, a principio o mais baixo, depois o do meio ou centro, por fim o alto. Os rios foram os caminhos que tiveram. Quando chegaram ás suas cabeceiras e as grandes vastidões territoriaes ainda se estendiam por ellas afóra,  começaram a abrir trilhos que os pusessem em communicação com os rios de curso mais longo que os contornavam alem, ou mais facilmente unissem os logares que melhor se  affeiçoassem ao genero de sua actividade, agrícola, pastoril ou mineira. Surgem as primeiras aldeias de naturaes do paiz sob a direcção espiritual do frade, apprecem as mais antigas fazendas de criar, despontam os sítios da lavoura. Ao lado do frade duas figuras de grande relevo, como que representando as demais, se levantam.

São Garcia d'Avila e Antonio Guedes de Britto,  os dois maiores proprietarios de terras na capitania da Bahia, troncos respectivamente das casas da Torre e da Ponte. Os seus curraes se enchem de gado, como se enchem os curraes de muitos outros sertanistas de destaque, e vê-se florescerem as aldeias de S. Antonio (Aratuhype), ao sul, Tatuapara, S. Pedro e S. André, ao norte.

Quando em 1624,cento e vinte quatro annos depois da descoberta do Brazil e setenta e seis da fundação da cidade do Salvador, Bahia de todos og Santos, os hollandezes quizeram fazer da terta bahiana sua preza, já o nosso littoral estava cortado de estradas e cheio de aldeias largamente povoados de indios pacificados e pelo baixo sertão ostentavam-se innumeros estabelecimentos agricolas e pastoris.

Tão notavel era já a obra do povoamento, que a esse tempo se descriminavam no reconcavo diversas freguezias, como fossem N. S, da Purificação de Santo Amaro, S. Bartholomêo de Pirajá,  N. S. da Conceição de Itapuan, Bom Jesus da Veneravel Cruz de Itaparica, N. S. da Piedade de Matuim e N. S. da Ajuda de Jaguaripe. Eram logares todos esses fartamente povoados. O sertão que até então se resentira
 desse beneficio, infestado que ainda se achava de indigenas refractarios a toda e qualquer tentativa de cathechese e pacificação. Mas ainda assim,  com poderoso auxilio de naturaes  do paiz, que se haviam deixado domesticar, não foi difficil resistir aos invasores, que aliás somente trinta annos  depois, em 1654, foram de todo expulsos.
 
 A luta retardou por muito tempo a obra da colonisação, mas não  a estagnou de todo. Durante ella vio-se surgirem novas aldeias, missão da Saúde, hoje cidade  de Itapicurú, S.S. Trindade de Massacará e outras. Uma vez terminada porem, redobraram-se os esforços e, já não havendo inimigos externos a combater, cuidou-se exclusivamente do inimigo interno, o indio, que senhor ainda de todo o sertão, trazia em sobresaltos as nascentes povoações, que, por elles atacadas inopinadamente, eram muitas vezes devastadas ou destruidas.
 
 A luta agora não foi menos cyclopica. Organisaram-se bandeiras, repetiram-se as entradas, abriram-se  estradas e vencidos definitivamente em 1693 os ultimos recalcitrantes, os Maracás, ficou firmado o domínio dos portuguezes, que desde então passaram a  gozar com toda tranquilidade as terras que prodigamente lhes foram dadas em sesmaria, os curraes que nellas estabeleceram, as fazendas que fundaram, os sitios  que estabeleceram. Criavam-se as primeiras villas, Jaguaripe em  1697, Cachoeira e S. Francisco de Sergipe do Conde em 1698. Affloraram diversas missões no sertão,
  N. S. das Neves (Sahy)\footnote{Atual Senhor do Bonfim-BA} % 				Senhor do Bonfim-BA
 em 1697, N. S. do "O" (Sorobabé)\footnote{Atual Ilha Surubabel em Rodelas-BA}%		Zorobabel ou Surubabel, em Rodelas-BA
 , S.  Francisco (Curral dos Bois\footnote{Atual Gloria-BA})%		Gloria - BA 
  em 1702, N. S. da Piedade (Hunhunhús)\footnote{Atual Santa Maria da Boa Vista - PE} % Sta Maria da Boa Vista - PE
  Bom Jesus (Jacobina), %		 			Jacobina
  N. S. dos Remédios (Pontal)\footnote{Ilha do Pontal, BA/PE} % 			Ilha do Pontal BA/PE (Juazeiro)
  em 1705, N. S. de Brotas (Joazeiro) %	Juazeiro - BA
 em 1706. O povo, descansado dos  hollandezes e já sem se encommodar muito com os indeios, começou a espalhar-se por todo o sertão, em procura de mineraes ou de logares onde  mais folgadamente podesse entregar-se a exploração de industria pastoril.
 
 Agua Fria e  Inhambupe tomaram impulso e fizeram-se freguezias, em 1718. Descobriram-se por esse tempo as minas de ouro de Jacobina e Rio de Contas. Para ellas affluiram mineradores. Formaram-se logo, ao seu redor, sitios de Lavoura, Coqueiros, Lagôa, etc.
 
 A noticia chegou celere aos ouvidos dos  homens do governo. Tratou-se de criar  villas para amparar o fisco. Um desembargador foi destacado para esse fim: adoeceu em caminho, ahi pelo sertão dos Tocós, e voltou. Não foi possivel que um outro se  incumbisse dessa tarefa. Lembrou-se então o governador do coronel Pedro Barbosa Leal, filho do celebre sertanista Francisco Barbosa Leal, sertanista como  seu pae, e foram fundadas as villas de Jacobina em 1720, e Rio de Contas em 1724, as primeiras villas do sertão bahiano, e ligadas por uma estrada. Mas as suas  situações  não agradaram. Em 1724 mudou-se a de Jacobina  da Missão do Sahy, onde ficava, para a Missão do Bom Jesus,  onde ainda hoje está, e em 1725 a do Rio de Contas da margem do Bromado, onde se levantava o planalto  onde desde então florece.
 
 \newpage Ligadas as duas villas entre si, foram logo unidas á cidade do Salvador, Bahia de todos os Santos, por estradas traçadas por cima das velhas picadas, dos trilhos primitivos. O surto nunca mais foi detido.
 
 Uma dessas  estradas, aberta por Garcia d'Avila e outros,  grande criadores de gado no alto sertão, entre os  annos de 1654 e 1698, para conducção de suas boiadas e rectificada e melhorada pelo coronel Pedro  Barbosa Leal em 1729, quando fundou a villa de S. Antonio de Jacobina,  cortava o sertão  dos Tocós, também chamado de Pindá, onde ficavam o arraial de  Agua Fria e as fazendas de Sacco do Moura, Serrinha, Tambatá, Massaranduba, Pindá, Cuyaté, etc. Em Serrinha, tomava ás direitas, pela fazenda  Raso, hoje villa Aracy, para Geremoabo e Pontal no rio S. Francisco, e  no tanque do Papagaio, adiante de Cuyaté, tomava ás direitas para Tiuba, ou Itiuba, como se diz hoje, e Joazeiro, no rio S. Francisco, as esquerdas para Jacobina. Serrinha, que em 1716 era simples logradouro da fazenda Tambutá, onde já morava Bernardo da Silva, passou em 1723 por compra que de suas terras e das terras do Tambuatá fez Bernardo da Silva aos antepassados da casa da Ponte, á sede da  fazenda, que deixou de ser Tambuatá para ser Serrinha, ficando Tambuatá como logradouro. Morto Bernardo, foi partilhada entre os seus herdeiros, que doaram algumas braças de terra, no local da casa da fazenda, a N. S. de Sant'Anna e ahi erigiram uma capella, que, concluida em 1780, foi em 1838, deseseis annos depois da  nossa emancipação  politica, elevada á cathegoria da  matriz da freguezia ahi então criada. A freguezia ficou annexada ao Termo da villa  de Purificação dos Campos, criada no anno anterior em substituição á villa de S. João Baptista de Agua Fria, então extincta, e só em 1876, por influencia e prestigio do inesquecivel capitão José  Joquim de Araujo, o popularissimo capitão Zezinho da Soledade, pae do autor destas linhas, gozou dos fóros de villa, e agora elevada a cidade.
 
 

