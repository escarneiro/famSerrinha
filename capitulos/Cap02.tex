%\chapterimage{pano-5.jpg} % Chapter heading image
\chapter{O Sertão dos Tocós}

 Em 1609 a penetração, que se havia encaminhado pelos rios - vias únicas de comunicação que então se podiam utilisar, inexistentes que eram as estradas - e atingindo as cabeceiras dos de curso menor, Joannes, Pojuca, Sabahuma, Inhambupe e Real, ganhou maiores extensões nos de maislonga carreira, Jequiriçá, Paraguassú, Itapicuru e São Francisco, principalmente São Francisco, traduzindo-se este facto em novas e mais vantajosas sesmarias, concedidas áquellesque nas conquistas arriscavam sua vida, fazenda e commodidades.
 
 O mestre de campo Antonio Guedes de Britto, bisavô  de  D. João de Saldanha, conde da Ponte, sertanista como seu pae Antonio Correia de Britto - o qual já havia tido muitas sesmarias e posteriormente  teria de haver várias outras mais importantes - por carta régia de 21 de Julho deste anno, 1609, obteve todas as terras existentes entre os rios Itapicuru e Inhambupe, nas cabeceiras das que já possuia nos nascentes dos rios Real  e Piagoay, e, para o sertão, mais dez léguas  medidas rumo direito com todas as pontas, enseadas, mattos, aguas e mais pertences (Felisbello Freire, Hist. Territorial do Brazil, pag. 28).
 
 Essa data  comprehende os então conhecidos por sertões dos Tocós, do Pindá  e do Tucano. Abrange os actuaes municipios de Queimadas, Tucano, Aracy (Raso), Coité, Serrinha e Riachão do Jacuípe, que, separados por pequenas distâncias entre si, se ligavam pela comunidade de interesses dos seus habitantes.
 
 Teve voga no seu tempo esta quadra, de antiguidade quasi secular, demonstrativa não só do espírito folgasão  de quem a compoz, senão ainda do desapontamento deparado  por quem esperava talvez somente goso bem estar nas incontinencias de um temperamento pronunciadamente andêjo, em displicencia por esses logares.
 \begin{verse}
 	\hspace{4em}Serrinha não Serra pao grosso,\\
 	\hspace{4em}Coité não dá selamim,\\
 	\hspace{4em}Raso não tem fundura,\\
	\hspace{4em}Queimadas não nasce capim.\\
 \end{verse}

 
 É uma zona, cuja fertilidade incontestavel é entretanto altamente prejudicada pela escassez dos rios perennes, pela inconstancia das chuvas, pelo flagelo das secas periodicas e exterminadoras, e que aliás foi promptamente povoada não só por ser trajeto obrigado da estrada da Bahia ao S. Francisco e ao Piauhy, por onde desciam as boiadas, como também por se prestar, na época, admiravelmente, á criação, hoje desfavorecida pelo augmento da lavoura.
 
 Doada, como todas as outras terras , com a clausula de occupação e cultivo, em prazo mais ou menos curto, o seu  donatário de espaço a espaço, com intermittencia de uma, duas, tres  e mais legoas, segundo a maior ou menos feracidade  dos terrenos e a maior ou menor possibilidade de captação das aguas das chuvas, fazia um curral,  ponha-lhe ao lado uma casinha e um cercado, ahi collocava uma familia de agricultores, facilitando ao seu chefe tudo, o escravo para o trabalho e o gado para criar, e cobrando-lhe renda modicissima, que raramente ultrapassava doze  mil réis e não pouco baixava a quinhentos réis por anno. E assim tinha fundado um  estabelecimento agrícola e pastoril, tinha feito um sítio. O rendeiro tomava conta desse sitio na esperança  de fazer-se proprietário delle por compra, e isto foi o que sempre aconteceu, se não logo, de certo mais tarde, quando já não  foram de temer nem as invasões dos estrangeiros - que demandavam homens e armas para defeza do paiz, nem os ataque sdos indios - que embaraçavam o desenvolvimento dos estabelecimentos agricolas.
 
 Encerrado o cyclo daquelas, posto termo a estes, o sentimento de propriedade despertou e o rendeiro, ja aproveitando melhor o seu  trabalho, poude fazer-se, por compra, senhor do solo, quasi todo em poder das casas influentes e poderosas.
 
 O mais antigo documento de alienação de Terras no sertão dos Tocós, de que temos noticia, é de 1716. É uma escriptura  publica de venda, que d. Izabel Maria Guedes de Britto, viúva do coronel Antonio da Silva Pimentel, passou na cidade do Salvador, Bahia de todos os Santos, em 31 de Maio, ao capitão Antonio Homem da Affosenca Correia,  morador nos campos do termo  da villa de Cachoeiram dos sitios  Massaranduba, Serra  Grande e Dois Irmãos, herança do seu fallecido pae Antonio Guedes de Britto, por 1:500\$000\footnote{A moeda do tempo era o "réis". Leia-se aqui o valor de "mil e quinhentos mil-réis" (ou mirréis)}
 
 A divisa desses sitios era a seguinte: começava
 onde fazia meio certo a estrada da fazenda Massaranduba,
 onde estavam as casas do capitão João Alvares Filgueiras\label{filgueiras}, para o “Tambuatá, onde morava Bernardo da Silva, e dahi, como pião, corria ramo direito para a parte do Nascente até chegar ao Morrinho que está entre o Saco Grande e a fazenda da Serra e Serrinha e elle corria direito á nascença do riacho da Tapera
 e por elle abaixo, com todas as voltas e e enseadas até as Capueiras que fizeram no rio Salgado, que, sendo o mesmo da dita Papera, em sua nascença lhe chamam Salgado, embaixo, nas capueiras, cincoenta braças abaixo dellas, seguia para a parte do Norte, rumo direito á quarta parte do norte, buscando as
 catingas até endireitar ou emparelhar com a casa do sitio chamado Dois Irmãos, e dahi seguia para diante com o mesmo rumo até se encher da meia legua e nesta forma se dividia então por esta parte, ficando bem como o sitio do Salgado, aonde morava Gaspar Pinto, para se medirem e demarcarem pelas partes
 do Poente. Tornava ao primeiro pião da estrada, que estava entre o Tambuatá e Massaranduba, e delle corria rumo direito para a parte do Poente até ficarem de dentro, para a parte da Massaranduba, do dito rumo à chegar á primeira baixa, vindo do riacho de Subaé, á mão esquerda, pela estrada que vinha para à Massaranduba e depois de salvar esta dita legua ia seguindo o mesmo rumo em que viesse até encher a meia do rumo que corria para as catingas com que se dividiam com o sitio Dois Irmãos e de um e outro rumo, onde findassem, se botaria o travessão pela parte da catinga.
 
 Só sete annos depois, em 1723,por escriptura publica passada na Bahia e pousadas de D. João de Mascarenhas, este e sua mulher d. Joanna da Silva Guedes de Britto, em notas do tabelião Manoel Affonso da Costa, venderam a Bernardo da Silva, morador no sertão dos Tocós por 2:200\$000, á vista, as terras do sertão dos Tocós e nellas um sítio chamado Serrinha, que houveram por titulo de herança de seu pae e sogro o coronel Antonio da Silva Pimentel, o qual sítio de terra, assim chamado a Serrinha, confronta e demarca por uma parte com terras delles vendedores, e sitio em que está de renda Gaspar Pinto, buscando o taboleiro que vae para a catinga onde morou Antonio Gonçalves e demarcando ao meio com o Saco do Moura, donde corre rumo à tapera do Cypriano, de lá cortando á entestar e demarcar com terras de Francisco de Sá Peixoto, de outra parte com rumo á partir com terras do cel. Antonio Homem, por outra parte ainda,corre direito o rumo à partir com terras de Manoel Carlos Lima, e deste corre rumo direito à entestar e partir com o Saco dos Tapuyo buscando a lagoa do Genipapo, correndo o rumo
 direito a entestar com terras do dito Francisco
 de Sá Peixoto.
 
 Eis  que era o sertão dos Tocós em 1723. Uma porção de sitios de lavoura e criação, a pequena distancia uns dos outros, um dos quaes Serrinha, hoje cidadem por um lado se limitavacom os sitios Massaranduba, Serra Grande e Dois Irmãos, de propriedade do coronel Antonio Homem da Fonseca Correia, que aliás não morava nelles, por outro lado com o sitio
 Salgado, propriedade de D. João Magicarenhas
 mulher d. Joanna, que o tinha arrendado a Gaspar Pinto, por outro lado com o sitio de Manoel Carlos de Lima e finalmente por outro lado com o sitio de Francisco Ge Sá Peixoto.
 
 Foi, pois, Bernardo da Silva, depois do mestre de campo Antonio Guedes de Britto, seu, genro coronel Antonio da Silva Pimentel e d. Joanna da Silva Guedes de Britto, flha deste casada
 em primeiras nupcias com D. João Mascarenhas e em segundas com D. Manoel de Saldanha, senhores primeiros de todo o sertão dos Tocós e de grandes tratos de terras nos sertões do Tucano, rio de S. Francisco, Jacobina, Itiuba, Rio de contas e Rio Pardo -  o primeiro e unico proprietario do sitio Serrinha, que, por sua
 morte passou a seus filhos e genros e hoje se acha na posse e domínio de seus tetranetos e filhos destes.
 
 Conforme se vê da escriptura de venda dos sitios Massaranduba. Serra Grande, Dois Irmãos, passada em 1716 por D. João de Mascarenhas e sua mulher D. Joanna ao capitão Antonio Homem, ddesde esse tempo Bernardo da Silva era morador no Tambuatá, terras do sitio Serrinha, como o capitão João Alvares Filgueiras o era em Massaranduba, onde tinha casas, e Gaspar Pinto no Salgado.
 
 Bernardo da Silva, portanto, sete annos antes de comprar o sitio Serrinha, já morava nelle logar Tambuatá, pagando renda,
 e todo esse sertão, desde 1698, segundo se vê de uma representação dirigida ao governo portuguez, estava povoado de moradores brancos com suas fazendas. (Inv. dos doc. rel ao Brazil no arch. de Mar. e Ultramar de Lisbõa, org. para a Bib. Nac. por Eduardo de Castro e Almeida, pag. 21)
 
 Então o caminho de Bahia a Jacobina e S. Francisco, por Agua Fria, até 18 ou 20 legoas de Bahia, tinha 3 freguezias: Rosario do Porto da Cachoeira, S. Gonçalo dos Campos e São José das Itapororocas, — e duas capellas,— N. S. da Conceição e N. S. do Desterro, - com suas igrejas e d'ahi pra cima, passando embora pelas povoações dos Tocós e do Pindá com bastante moradores, não tinha uma só Igreja, uma só freguezia, mas tão somente uma capella curada em Jacobina.
 
 Em 1731, oito dam depois da compra de Serrinha por Bernardo da Silva e trinta e três da representação a que vimos de nos referir, quando o mestre de campo Joaquim Quaresma Delgado, encarregado por portaria de 11 de Janeiro de percorrer os districtos de mineração da capitania da Bahia escreveu os seus celebres roteiros de Bahia a Jacobina, de Jacobina a Minas do Rio de Contas e de Minas a Bahia, a situação desses sertões só havia mudado em se haver criado em 1718 uma freguezia em
 Agua Fria e ahi erigido uma igreja e em 1727, com o pelourinho e a força, installado uma vila.
