\chapter{O Tronco da Família}

Em sua rota de Passagem ou Trapiche Velho, nos arredores da cidade do Salvador, Bahia de todos os Santos, a Jacobina, o mestre  de campo Joaquim Quaresma Delgado, palmilhara a estrada das boiadas, cujo traço foi depois seguido pela ferrovia da Bahia a Joazeiro. Pequenos ranchos, poucos moradores, povoados insignificantes, foi tudo quanto encontrou. Em todo seu peregrinar só vio um engenho de assucar. Pojuca foi o lugar venturoso que o ostentou. Em todos os outros, alguns dos quaes hoje vilas mais ou menos importantes como Matta de São João, Catu e mesmo Pojuca, cidades de muita actividade e progresso, por exemplo, Alagoinhas e Serrinha, não notou senão ranchos, casas raras e e diminutissimo numero de habitantes. Faziam excepção a esta regra, Agua Fria com a sua Igreja de S. João Baptista de Agua Fria e as mais casas distantes da estrada a parte do oeste coisa de meia legua. Serrinha logar de bastantes casas com seus moradores e fazenda de gado, e Jacobina, villa de uma rua arruada leste oeste, com casas de uma banda e da outra.  Jacobina foi villa em 1724. Agua Fria já era freguezia desde 1718, e villa á partir de 1727, quando a carta regia de 24 de Abril a elevou a essa dignidade, desmembrando-a de Cachoeira, que foi creada villa em 1689. Serrinha lhe pertencia, mas como uma simples fazenda de gado, de propriedade de Bernardo da Silva, criador abastado e chefe de numerosa prole. Só por morte de Bernardo, a qual devia ter ocorrido pouco antes de 1763, senão n'este anno, é que se pensou em érigil-a em Arraial, embora já tivesse, desde muito tempo, a sua capella sob a invocação de Senhora Sant'Anna.

Em um documento de 1763 se fallou no "sitio de Senhora Sant'Anna de Serrinha"; em outro de 1775 se fallou tão somente "sitio Serrinha", mas em 1814 já se fala, no "arraial da Serrinha". Serrinha, portanto só foi arraial depois de 1750, quando foi concluida a capella, que ainda hoje é a Matriz da freguezia, O arraial conseguintemente não é obra de Bernardo; mas de seus filhos e genros. Foram estes senão os iniciadores, pelo menos os concluidores da construção da igreja, que ainda hoje é a mais sumptuosa das suas construções, toda de pedra e cal que é. É crível que o impulso poderoso tivesse partido de sua viuva, Maria do Sacramente, fortemente amparada por seus filhos; Pe.Prudente e Maria da Purificação, esta solteira, d'uma piedade rara, de uma vida sem um  minimo deslise, toda virtude e bondade. Nenhuma duvida existe sobre a personalidade de Bernardo da Silva, de sua mulher Maria do Sacramento e seus filhos. Eram elles em 1731 os moradores das "bastantes casas", bem como os proprietários da fazenda de gado, que o mestre de campo Quaresma, de viagem de Bahia para a villa de Jacobina, encontrou em Serrinha. Não sei se elles eram portuguezes, ou tão somente descendentes de portuguezes.

Talvez Bernardo da Silva, seja filho, ou mesmo neto, de Sebastião. da Silva, tabellião na cidade do Salvador, Bahia de Todos os Santos, que de 1612 a 1653 era proprietario de terras entre os rios Sabauma e Inhambupe, e de quem eram descendentes Roberto da Silva e outros que ainda em 1761 eram heréos confinantes do fidalgo Manoel de Saldanha, senhor de grandes tratos de terras no sertão dos Tocós, e de cujos antepassados Bernardo da Silva houve por compra, em 1723, o sitio Serrinha. Seja ou não, elle era entretanto de fina linhagem portugaesa, e dos honrados portuguezes d'aquella época a tempera inamolgavel de finissimo aço.

Bernardo da Silva, que em 1716 morava no Tambuatá, em 1723 comprou o sitio que recebeu o nome de Serrinha, e terras de Tambuatá era, e ahi se firmou, fazendo casa, abrindo tanque, o ainda hoje conhecido por tanque da Nação, cuidando da criação, que era a preoccupação das maiores activilades de então. Por escriptura peblica de 7 de Dezembro de 1737, seis annos depois da passagem de Quaresma por Serrinha, Bernardo da Silva comprou a Domingos Garcia de Aragão, representado por seu procurador Manoel Gomes do Rosario, constituido por instrumento passado em Iguape em seis do mesmo mez e anno, a fazenda Sacco do Moura, que, em execução movida contra Francisco Ribeiro Pinto, arrematara em hasta publica no termo de Agua Fria.

A compra foi feita por 3:400\$000 comprehendendo 500 cabeças de gado e a escriptura foi passada pelo tabelião da villa de Nossa Senhora do Rosario do Porto da Cachoeira com as divisas seguintes: da casa para cima com terras de Serrinha, delle comprador; pela parte do norte, por um riacho abaixo chamado a Pipoca; e da casa para baixo, buscando o Lamarão, partindo com o primeiro riacho que stá adiante do Lamarão chamado Umbarana correndo direito a buscar o dito riacho da Pipoca; e pela parte do sul, correndo rumo direito a buscar a matta dos Tocós e pela dita matta acima passando a capoeira chamada da Izabel a buscar o riacho da Tapera do Cypriano. Em 1763, vinte e seis annos depois da compra do sitio Sacco do Moura por Bernardo Silva, este já era morto, havendo lhe sobrevivido a sua mulher Maria do Sacramento, que a esse tempo gozava bôa saude em sua fazenda Serrinha, onde morava com alguns de seus filhos e genros. De facto, nesse anno de 1763 e dia 24 do mez de Outubro, no sítio de Senhora Sant'Anna de Serrinha, Termo da villa de S. João Baptista de Agua Fria, e casa de morada da senhora Maria do Sacramento viuva do defuncto Bernardo da Silva, o reverendo Padre Prudente da Silva, e Fructuoso de Oliveira Maya e sua mulher Bernarda Maria, o alferes José da Silva e Oliveira e sua mulher Ana de Jesus e Silva, e Maria da Purificação por escriptura publica passada por Manoel Jorge Coimbra, tabelião da villa de Agua Fria, venderam ao seu irmão e cunhado o capitão Apollinario da Silva, as partes que tinham na fazenda Sacco do Moura, no valor parcial de  171\$428 e  total de 685\$712, que lhes  tocaram no inventario feito no Juizo de fallecimento do defuncto seu pae e sogro Bernardo da Silva. Não devia haver decorrido muito tempo da morte de Bernardo, primeiro proprietario do sitio, hoje cidade de Serrinha, comprado à casa da Ponte, e tronco da familia de Serrinha.

Por este documento se vê que cinco filhos tinha elle de certo a saber: o Padre Prudente, o capitão Apollinario, o alferes José da Silva Oliveira, Bernarda Maria casada com Fructuoso de Olivelra Maya e Maria da Purificação, que não se casou e diz um velho manuscripto que "viveu sempre na casa paterna, e foi de uma reputação illibada". Havemos de vêr que outros teve elle como sejam Ignacio Manoel da Silva, do Genipapo, e mais quatro filhas casadas respectivamente com Antonio Carneiro da Silva, de S. Bartholomeu, Antonio Manoel da Motta, do Tambuatá, Domingos  Ferreira Santiago, de Serra Grande, e Miguel Affonso Ribeiro, do Sitio.

Antonio Carneiro da Silva e Miguel Affonso Ribeiro, bem como o capitão Apollinario da Silva, foram testemunhas instrumentarias da escriptura publica de venda de partes das terras do sitio Candeal, que em notas do tabelião da vila de Agua Fria, Antonio Pinto dos Reis, Fructuoso de Oliveira Maya e mulher Bernarda Maria da Silva, seus proprietarios, moradores no sítio Serrinha, passaram em 18 de Dezembro de 1775 ao alferes José da Silva Oliveira.

Antonio Carneiro morava em S. Bartholomeu, hoje Termo do Riachão de Jacuhype e então de Cachoeira; Apollinaria da Silva residia no Sacco do Moura e Miguel Affonso Ribeiro tinha seu lar no sitio mais tarde conhecido por sitio dos Carneiros, por haver ahi fixado residenciao alferes José da Silva Carneiro, filho de Antonio Carneiro da Silva e genro do dito Miguel Affonso Ribeiro, e se tornado chefe da numerosa prole, toda com o sobrenome de Carneiro.

Dois filhos de Bernardo, o Padre Prudente e Maria da Purificação, não constituiram familia e portanto não tiveram descendencia, os demais constituiram-na e delles procedem as familias de Serrinha, Sacco do Moura, Serra Grande, Tambuatá, Sitio, S. Bartholomeu, Genipapo e Tiririca, que por sua origem, por seus entrelaçamentos, constituem a familia de Serrinha.

Essas familias ramificaram-se depois por Cachoeira, Feira de Sant'Anna, S, José de Itapororocas, Santo Amaro, Agua Fria, Pedrão, Inhambupe, Riachão do Jacuhype, Coité, Monte Alegre, Jacobina, Sento Sé,  etc.

Para cada uma dellas terei um capítulo especial, ligado que estou pelo lado paterno familias de Tambuatá e Genipapo, e pelo materno á de Serrinha, ou a todas pelo sangue commum de Bernardo da Silva, o esquecido antepassado, cuja memoria fariamos bem, em relaçar, erguendo-lhe um monumento no logrdouro mais importante do logar, ou tão apenas a elle dando-lhe o nome venerando, e digno de melhor acatamento por parte dos seus descendentes, que são todos os da élite Serrinhense.