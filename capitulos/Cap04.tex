\chapter{Os Apolinários}

\begin{centering}
	\section*{Familia do Sacco do Moura}
\end{centering}

Foi o capitão Apollinario da Silva o mais representativo dos filhos e genros de Bernardo
da Silva e sua mulher Maria do Sacramento.

Casando-se, segundo se conta, com uma senhorita da familia Trindade, da freguezia de S. José das Itapororocas, estabeleceu-se na fazenda Sacco do Moura, que pertencia a seu pae e por morte deste passou a lhe pertencer, a titulo de herança e de compra dos quinhões hereditarios de seus irmãos e cunhados. A compra de alguns desses quinhões, os do Padre Prudente da Silva, de Fructuoso de Oliveira Maya e sua mulher Bermanda Maria, os alferes José da Silva e sua mulher Anna de jesus e Silva e o de Maria da Purificação,como já tive occasião de escrever, foi feita por escriptura publica de 24 de Outubro de 1763. Em1817 o capitão Apollinario era morto, havia algum tempo, mas em 1796 elle era vivo e em 17 de Outubro desse anno foi constituido procurador por Manoel Telles de Almeida, morador na sua fazenda Rio do Peixe, no julgado da serra de Itiuba, conjunctamente com o sargento mór Manoel de Jesus Maria e Antonio José de Mello, para assignar a escriptura de compra da fazenda Carahibas, no sertão dos Tocós, cuja escriptura foi passada por d. Francisca Joanna Josepha da Camara, viuva de Manoel de Saldanha, na Bahia e cartorio de tabellião João Damasio José, representada por seu procurador capitão Manoel Pinto da Cunha e assignada por Antonio José de Mello, como procurador do comprador.

O capitão Apollinario, teve os seguintes filhos: Josepha que se não casou e por haver se tornado prodiga, dissipando a herança paterna, foi  declarada interdicta, judicialmente, sendo os seus bens vendidos em hasta publica em 20 de Janeiro de 1817 e artematada por 73\$500 pelo sargento mór Manoel de Jesus da Silva Gomes a parte que tinha na fazenda Sacco do Moura e era enconstada ao dito sargento mór.

Ignez, que se casou com Manoel de Jesus, natural da freguezia de S. Pedro do Rio Fundo e estabeleceu-se no engenho Gravatá.

Uma que se casou com Francisco Cordeiro, portuguez, e se estabeleceu no sítio Lamarão, hoje povoado de alguma importancia e movimentada estação da estrada de ferro S. Francisco, abaixo de Serrinha quatro leguas. Uma que se casou com Antonio Joaquim, portuguez, que foi enforcado por um seu escravo alfaiate.

Uma, que se casou com Luiz Antonio Vieira, natural da capital e morador na fazenda Sacco do Moura.

Parece que a  filha casada com Antonio Joaquim não deixou descendencia. Pelo menos, não é ella conhecida. A filha casada com Francisco Cordeiro, deixou dois filhos, a saber, Pedro José Cordeiro e Joanna, que se casou em Primeiras nupcias  com Ignacio José de Medeiros e fundou a fazenda Soleira e, em  segunda nupcias, com João Manoel de Freitas. Maiores foram as descendencias das filhas casadas respectivamente com Manoel de Jesus e com Luiz Antonio Vieira. E  nãoforam  tão somente maiores, foram ainda mais distinctas. Questão de meio e só e só de meio. Collocando-se no Reconcavo, tiveram ensejo de  melhor aproveitar e fazerem prosperar os seus bens de fortuna e a sua actividade.

Foram filhos de Manoel de Jesus e sua mulher Ignez, filha do capitão Apollinario:  Padre José Appolinario de Gouveia; Tenente Manoel de Jesus Gouveia, que casou com sua prima Anna Maria, filha  de seu tio José Affonso Ribeiro; Anna, que não se casou; Uma que casou com Manoel Ferreira Lustosa, do engenho Brejões, e não  teve filhos; Maria, que foi  casada com seu primo José Apollinario Vieira, do engenho Brejinho; Thereza, que foi casada com Francisco da Silva Mello, do engenho Orobó; Ignez, casada com José Moreira de Carvalho, do engenho Agua Bôa.

Os filhos de Luiz Antonio Vieira, casado que foi com uma filha do capitão Apollinario Silva, são: José Apollinario, Vieira, casado com Maria, filha Manoel de Jesus; Joaquim José Vieira que se casou com Anna Cardoso, sua prima, filha do tenente Manoel de Jesús Gouveia; Manoel José Vieira, que casou com Maria da Representação, filha do tenente José da Silva Carneiro e sua mulher Anna Moreira.


Agora vejamos os tetranetos de Bernardo da Silvas e sua mulher Maria do Sacramento, fundadores e primeiros proprietarios da fazenda Serrinha, pelo ramo dos Apollinarios.

Tetranetos de Bernardo da Silva, bisnetos do capitão Apollinario da Silva, netos de Manoel de Jesús, filhos do tenente Manoel Jesús Gouveia: \textbf{I}, Ignez, que se não casou; \textbf{II}, Rita,
quee não casou; \textbf{III}, Francisca, que não casou; \textbf{IV}, Anna Cardoso,. que casou com Joaquim José Vieira, e teve os seguintes filhos: padre Jósé Joaquim Vieira, bacharel, Luis Antonio Vieira, juiz de direito aposentado do Estado da Bahia, dr. Francisco Joaquim Vieira medico, que se casou com d. Luiza de Freitas Barros, de Oliveira de Campinhos; Manoel José Vieira, casado com d. Amelia de Almeida, Joaquim José Vieira, casado com d. Theodora Moreira de Pinho, Ignez casada com Antonio Dantas de Souza, Anna Maria, casada com seu primo Manoel José Vieira.

Tetranetos de Berrnardo da Silva, bisnetos do capitão Apollinario da Silva, netos de Manoel de Jesús, filhos de José Apollinario Vieira, casado com uma filha de Manoel de Jesús: Ignez, que se casou com o Barão de Pojuca, de cujo consorcio teve uma filha, que se casou com o almirante Joaquim Pinheiro de Vasconcellos.

Tetranetos de Bernardo da Silva, bisnetos do capitão Apollinario da Silva, netos de Manoel de Jesús, filhos da filha casada com Francisco da Silva Mello: \textbf{I}. Francisco da Silva Mello Junior, casado com D. Marianna; \textbf{II}. Ignez, casada em primeiras nupcias com o major Honorato Guimarães Leal, de cujo casamento tiveram dois filhos, sendo que delles uma filha casou com José Lopes de Carvalho.

Tetranetos de Bernardo da Silva, bisnetos do capitão Apollinario da Silva, netos de Manoel de Jesús, filhos da filha casada com José Moreira de Carvalho, do engenho Agua Bôa: textbf{I}. Marianna, casada com Francisco da Silva Mello, de cujo consorcio nasceram d. Thereza, com quemo dr.João Ferreira de Araujo Pinho, ex-governador da Bahia, se casou em primeiras nupcias e teve dois filhos, o dr. João Ferreira de Araujo Pinho Filho e d. Maria Pinho, ambos vivos; o dr. Francisco Moreira de Carvalho, Conde do Subahé fallecido em estado de solteiro; José Moreira de Carvalho; Ignez, casada com o major Honorato Moreira Rego,do engenho Paraaassú, que não deixou filhos.

Tetranetos de Bernardo da Silva, bisnetos do capitão Apollinario da Silva, netos de Luiz Antonio Vieira, filhos de José Apollinario Vieira; Ignez, casada com o Barão de Pojuca, ldem idem, filhos de Joaquim José Vieira e Anna Cardoso: \textbf{I}. Padre Jose J. Vieira, bacharel Luiz Antonio Vieira, de. Francisco Joaquim Vieira, Manoel José Vieira, Ignez e Anna Maria.

Tetranetos de Bernardo da Silva, bisnetos do capitão Apollinario da Silva, netos de Francisco Cordeiro, filhos de Joanna: \textbf{I}. Isidro que se não casou: \textbf{II}. André que se casou com uma senhora da Purificação dos Campos.(1ª nupcias com Ignacio de Medeiros;) \textbf{III}. Cesaria que casou com o tenente coronel Joaquim Carneiro de Campos (2ª nupcias com João Manoel de Freitas) e outros cujos nomes não tenho de memoria.

Si o capitão Apollinario da Silva foi o mais representativo dos filhos e genros de Bernardo da Silva, o dr. João Ferreira de Araujo Pinho é o mais illustre e o mais representativo dos descendentes do fundador de Serrinha, que são muitos mil. Quererá elle esta gloria? Não sei.

Devo declarar ter visto em poder do snr. Alfredo Vieira, do Sacco do Moura, alem dos documentos relativos a esta fazenda a que tenho me referido, uma escriptura, pela qual em 25 de Dezembro de 1834, José Evaristo da Silva e sua mulher Felippa Maria de Jesús venderam a Joaquim José Viera, juiz triennal em Agua Fria, por 40\$000, a parte de terras que tinham no Sacco do Moura, heran~a de seu pae e sogro Antonio da Silva Braga. Braga  será descendente de Antonio Joaquim, genro do capitão Apollinario? Ignoro.


