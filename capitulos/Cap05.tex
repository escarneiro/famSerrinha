
\chapter{Os Affonsos}

\begin{centering}
\section*{Familia do Sítio}
\end{centering}
Miguel Alonso Ribeiro, portuguez casado com uma filha de Bernardo da Silva e Maria do Sacramento, teve tres filhas, Uma casou com o alfere José da Silva Carneiro, seu primo, filho de Antonio Carneiro da Silva, concunhado de Miguel Affonso, Outra teve por marido a José Affonso, portuguez. A outra Anna Maria da Silva, consoriou-se com o capitão Manoel de Affonseca Pinheiro, também portuguez. Si teve filhos, não chegaram elles á maioridade, nem constituiram familia pelo casamento.


O alferes José da Siva Carneiro só teve dois filhos - o capitão José Carneiro da Silva, que casou com Maria Francisca da Purificação, sua prima, filha  de seu tio José Affonso, e Anna da Silva Carneiro,  que casou com Tristão Gomes da Silva, seu primo, neto como ella, de Bernardo da Silva e Maria do Sacramento, pelo ramo dos Mayas, do qual o cabeça foi Fructuoso de Oliveira Maya, genro de Bernardo.  Enviuvando o alferes José da Silva Carneiro passou a segunda nupcia com Anna Maria Moreira, filha do  capitão Manoel José Moreira, portuguez, bisneta de Bernardo, pelo mesmo ramo dos Mayas, de cuja cabeça, Fructuoso, o capitão Manoel José Moreira era genro; e tendo fallecido em 28 de janeiro de 1854 com mais de 90 annos, dizem que noventa e oito annos e tres meses — pois em 1784 já era alferes e seu pae, Antonio Carneiro da Silva, já ratificava nesse anno o dote que lhe havia feito da fazenda Tanque — deixou muitos filhos desse segundo consorcio, ligação de Carneiros e Mayas.

José Afonso teve os seguintes filhos: José Affonso Ribeiro, que casou com Anna Maria, sua prima, filha de Ignacia Maria da Silva, uma das filhas de Antonio Carneiro da Silva, e Matheus da Silva Cardozo vulgo Petéo, do Pedrão; Miguel Affonso Ribeiro, que em 1836 era juiz de paz em Serrinha, então elevada a freguezia, e era casado com Bernarda Maria da Silva, sua prima, filha dos sobredictos, Ignacia e Matheus da Silva Cardozo; Anna as que casou com seu primo Manoel de Jesús Gouveia, bisneto, como ella de Bernardo da Silva, pelo ramo dos Apollinarios, que teve por primeiro o capitão Apollinario Silva, filho de Bernardo; Anna Francisca, que casou com seu primo Matheus Carneiro da Silva, filho de Ignacia e Matheus da Silva Cardozo; Maria Francisca da Purificação, casada que foi com o capitão José Carneiro da Silva, filho do alferes José da Silva Carneiro, neto de Antonio Carneiro da Silva e bisneto, como ella de Bernardo da Silva; e Manoel Affonso Ribeiro, que em 1839, foi avaliador no inventario: de Maria Francisca, mulher de José Ramos. Manoel de Affonseca Pinheiro foi dos tres genros de Miguel Affonso Ribeiro, o que mais netos lhe deu.

Sete foram elles a saber: José Pinheiro da Fonseca, que não se casou, Custodia que foi casada com João Manoel da Motta, Anna Maria da Silva, que se casou com Antonio de Mattos Paim, viuvo, filho de d. Anna Innocencia da Silva, tambem viuva, natural do termo de Jacobina, onde era proprietaria dos sitios Umburana e Queimada dos Moços, que tinham sido da casa da Ponte e por elle, e sua segunda mulher, foram vendidos por escriptura publica de 13 de Novembro de 1821, passada em Jacobina, a Antonio Joaquim de Oliveira; Bernarda, que se casou com Vicente Ferreira da Silva, filho de Ignacio Goncalves Pereira, do Genipapo, e Thereza, filha de Ignacio Manoel da Silva, e neto de Bernardo da Silva; Antonia\label{ampaim}, que se casou com José Pereira Pinto, irmão de Vicente Ferreira da Silva, Maria Pinheiro, casada com o professor Antonio Martins Ferreira, tetraneto de Bernardo (ramo M. Santiagos) e pae dos padres Loreto mais netos lhe deu. 

Sete foram elles\footnote{O Autor falou em sete, mas aparentemente errou as contas ou esqueceu-se de dois.}, a saber: José Pinheiro da Fonseca, que não se casou; Custodia que foi casada com João Manoel da Motta; seu parente ramo dos Mottas; Anna que se casou com Antonio dos Mattos Paim, viuvo, natural do termo de Jacobina, onde era proprietario dos sítios Umburana e Queimada dos Moços, comprado a caza da Ponte, cujo sitio por escriptura publica de 13 de Novembro de 1821, passada em Jacobina, venderam a Antonio Joaquim dg Oliveira; pe. José Alves de Loreto e Urbano Cecilio Martins, o qual, tendo inviuvado, seus filhos, passou a segunda nupcia com Anna Francisca Carneiro filha do capitão José da Silva Carneiro a prima de sua primeira mulher, com quem teve aquelles dois filhos e mais três filhas.

Como se vê, por não haver Miguel Affonso tido filhos, mas tão semente filhas, o sobrenome de «Affonso» teria desappareçido logo na primeira geração, se não se casara uma das suas filhas com José Affonso, de certo seu parente, sobrinho talvez. Desapareceu porem nas imediatas. Substituiram-n'o os de Carneiro e Pinheiro. Até mesmo os netos de José Affonso não o conservaram, Retiveram somente o Ribeiro. E assim que os dois unicos filhas de José Affonso Ribeiro, netos do velho José Affonso, chamavam-se José Emigdio Ribeiro e Manoel Cardozo Ribeiro, recebendo este o Cardozo do seu avô materno, Matheus da Silva Cardozo.E a filha unica de Miguel Affonso, seu irmão, Maria, conhecida por Santinha, a qual se casou com o alferes Rozendo Carneiro da Silva, neto de Matheus da Silva Cardozo, não viu luzir em sua descendencia o sobrenome de Affonso, que outr'ora assignalava sentimentos de amor e fidelidade a velhos reis portuguezes desse nome.

Quem dirá que José Emigdio Ribeiro, José Carneiro da Silva e José Pinheiro da Fonseca por exemplo, sejam primos e primos carnaes? Pois o são.

O sobrenome nada indica, mas o parentesco entre elles é esse mesmo e maior não é a aproximação do sobrenome delles com seus demais primos. Seja esta razão para se lhes reconstituir a genealogia, triste que é dois parentes bem proximos se acharem em contacto e ignorarem  laços de parentesco que os prendem.

Conhecidos, pelo ramo dos Affonsos, os netos e bisnetos de Bernardo da Silva, façamos um quadro demonstrativo dos seus tetranetos.

Tetranetos de Bernardo da Silva, bisnetos de Miguel Affonso Ribeiro, netos de José Affonso, portuguez, filhos de José Affonso Ribeiro:
\textbf{I}. Tenente José Emigdio Ribeiro, que se casou com Anna Bernardina Moreira, filha de Manoel José Pinto, portuguez, e sua mulher Bernarda Archanja Moreira, ramo dos Mayas: moravam na fazenda Boa Sorte, terras da fazenda Sitio, hoje conhecido por Sitio dos Carneiros e tiveram muitos filhos, a saber, José Emigdio Carneiro, Anna Maria, Antonio Pinto Ribeiro, Maria Francisca, Manoel, Deoclecio e Modesto; \textbf{II}. João Cardozo Ribeiro, que, embora houvesse fallecido em idade bastante avançada, sempre se conservou em estado de solteiro.

Tetranetos de Bernardo da Silva, bisnetos de Miguel Alfonso Ribeiro, netos de José Affonso, filhos de Miguel Affonso Ribeiro: Maria, conhecida por Santinha, que se casou tom o alferes Rozendo Carneiro da Silva, seu primo, filho de Matheus Carneiro da Silva e neto de Matheus da Silva Cardozo: morava na villa e teve muitos filhos .

Tetranetos de Bernardo da Silva, bisnetos de Miguel Affonso Ribeiro, netos de Maroel de Affonseca Pinheiro, filhos de João Manoel da Motta: Tito Motta, que se casou com uma senhora do Reconcavo; Antonio Motta, que se casou em S.  Gonçalo dos Campos; Clementina, que se casou com José Dyonisio, de S. Gonçalo dos Campos; Maria, que se casou com o capitão Luiz Simões Ferreira, do Coração de Maria; Virginia, Carolina e Anna, que se não casaram.

Tetranetos de Bernardo da Silva, bisnetos de Miguel Afonso Ribeiro, netos de Munoel de Atfonseca Pinheiro, filhos de Antonio de Mattos Paim:

\hspace{2em}1º Maria Paim da Silva, morta cm 1843, que foi casada com Serafim Manoel da Motta, ramo dos Mottas (familia de Tambuatá): moravam na fazenda Marroaz e tiveram quatro filhos - Fernando, Anna, João e Maria.

\hspace{2em}2º Antonia, que se casou com à Manoel Pinheiro da Silva, filho de Vicente Ferreira da Silva e tetraneto de Bernardo da Silva, ramo dos Silvas (Ignacio Manoel da Silva).

Tetranetos de Bernardo da Silva, bisnetos de Miguel Affonso Ribeiro, netos de Manoel de Affonseca Pinheiro, filhos  Vicente Ferreira da Silva: \textbf{I}. Maria Vicencia (Lalia), que se não casou, \textbf{II}. Antonio Ferreira da Silva Pinheiro, que se casou com Virginia Ferreira Goes\label{vfgoes}, filha de José de Goes, e teve 2 filhos; Maria Magdalena\label{mmagdalena} da Posia e Jovina, \textbf{III}. Clementina da Silva Pinheiro, que se casou com Jose Antonio da Silva. \textbf{IV}. Constança da Silva Pinheiro, casada que foi com Manoel Pedreira Marques de Freitas, de cujo consorcio nasceram o coronel Leoncio Freitas, o dr. Graciliano Freitas, Estephanio, Genoveva casada com o coronel Marianno Silvio Ribeiro e Adelaide casada com o capitão Basilio Cordeiro. \textbf{V}. Francisca da Silva Pinheiro, que não se casou. \textbf{VI} Eduarda da Silva Pinheiro, que não se casou. \textbf{VII} José Pinheiro da Silva, que se casou com Antonia, filha de Antonio Mattos Paim. \textbf{VIII}. Jesuina,\label{jesuina} casada que foi com João Paes Cardozo. \textbf{IX}. Felisberto Ferreira da Silva, casado com Maria Freitas, irmã de Manoel Pedreira M.de Freitas. \textbf{X}. José Pinheiro da Silva.

Tetranetos de Bernardo da Silva, bisnetos de Miguel Affonso Ribeiro, netos de Manoel de Affonseca Pinheiro, filhos de José Pereira Pinto: Anna, que se casou com, Joaquim Bento de Souza; Virginia, que se casou com José Antonio da Silva; Maria Leopoldina, que se casou com o major João Pereira das Mercês; Geronyma, que se casou com Joaquim Cordeiro. Anna Rosenda, que se casou com Francisco Cordeiro.

Maria Pinheiro, casada com o professor Antonio Martins Ferreira, não deu tetranetos a Bernardo. Quanto aos filhos de Anna Maria, Anna Francisca e Maria Francisca da Purificação, casadas respectivamente com o tenente Manoel de Jesús Gouveia, com Matheus Carneiro da Silva e com o capitão José Carneiro da Silva, todos netos de José Affonso, bisnetos de Miguel Affonso Ribeiro, o velho, e tetranetos de Bernardo da Silva, encontrar-se-ão nos capitulos relativos a seus respectivos paes. Os filhos de Manoel de Jesús Gouveia, serão nomeados no ramo dos Apollinarios (familia do Sacco do Moura) e os de Matheus Carneiro da Silva e do capitão José Carneiro da Silva no dos Carneiro (familia de S. Bartholomêo).

