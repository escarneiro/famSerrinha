\chapter{Os Silva e Oliveira}
\begin{centering}
	\section*{Familia da Tiririca}
\end{centering}


O professor Antonio Martins Ferreira, filho de Serrinha, e dos mais ilustres, pae dos padres Loreto e Urbano, casado duas vezes em Serrinha, em ramos diferentes da familia, falecido, ha trinta annos mais ou menos, no Rio de Janeiro, para onde se mudou com toda a familia, quasi octogenario, senão octagenario, deixou um manuscripto intitulado Genealogia da Familia de Serrinha. Uma copia desse manuscripto, de seu proprio punho, existe em poder do meu irmão padre José de Cupertino e Araujo Lima, vigario de Sant'Anna do Catú, que o alargou consideravelmente, no que tem despendido para mais de trinta annos de esforços extraordinarios. Nesse documento se dá como mulher de Bernardo da Silva a Josepha Maria Silva. Mas há engano. Josepha, que foi interdictada judicialmente, por prodigalidade, em 1817, é filha do capitão Apollinario da Silva e neta de Bernardo da Silva. A mulher deste, que lhe sobreviveu, chamava-se Maria do Sacramento e isto consta da escriptura publica de compra de diversos quinhões hereditarios da fazenda Saco do Moura feita pelo capitão Apollinario a irmão e cunhados seus.

Nesse mesmo trabalho Francisco da Silva e Bernardo da Silva, aquelle estabelecido no sitio Tiririca e este, que não deixou descendentes, no sitio Sipó, figuram como filhos de Bernardo da Silva. Não figura como tal porem, o alferes José da Silva e Oliveira. É outro engano, a que estão sujeitos os trabalhos fundados exclusivamente em informações. O alferes José da Silva Oliveira, este sim, é filho, de Bernardo da Silva. Parece-me que Francisco da Silva e Bernardo da Silva são netos do velho Bernardo da Silva, o fundador de Serrinha. Filho garanto que o é o alferes José da Silva e Oliveira. Em 1775, por escriptura publica de 18  de Dezembro, os meus tetravós maternos Fructuoso de Oliveira Maya e sua mulher Bernarda Maria da Silva, senhores e
possuidores da fazenda ou sitio Candeal, venderam ao alferes José da Silva e Oliveira uma parte desse sitio, desmembrando-a nestes termos: As  terras são confrontadas pela parte do poente com terras delles vendedores, a saber, fazendo extremo na lagoa do Cançanção; e da dita lagôa, correndo rumo direito, pela parte do sul, a buscar a estrada que vae do Genipapo para  S. Bartholomeu, partindo com  as terras de Bom Sucesso pela serra do Macaco; e pela parte do Sul,partindo com terras delle comprador, da dita lagôa do Cançanção; pela parte do norte, fazendo pião na mesma lagôa, della correrá rumo direito de oeste ao leste até topar terras da Serrinha, com as quaes parte pelo lado nascente, como tambem por esse lado com as de S. Nicoláo.

Nesta escriptura passada pelo tabellião de S. João Baptista de Agua Fria, Antonio Pinto dos Reis, no sitio Serrinha, onde  moravam os vendedores, perante as testemunhas capitão
Apollinario da Silva, Miguel Affonso Ribeiro e Antonio Carneiro da Silva, não dizem Frutuoso e sua mulher si o alferes José da Silva e Oliveira é seu irmão e cunhado. Mas na escriptura passada em 24 de Outubro de 1763, doze annos antes, no sitio da Senhora S. Anna de Serrinha e casado morada da senhora Maria do Sacramento, viuva do defuncto Bernardo da Silva, pelo tabellião Manoel Jorge Coimbra, da compra de quinhões hereditarios na fazenda Sacco do Moura feita pelo capitão Apollinario a seus irmãos e cunhados, o alferes José da Silva e Oliveira e sua mulher Anna de Jesús e Silva figuram como vendedores e o capitão Apollinario, o comprador, como  seu irmão e cunhado,

e os quinhões vendidos todos eguaes, como parte de terras que os vendedores tinham no sitio Sacco do Moura, no qual tocou a cada um a quantia de 171\$428, no inventario que fizeram no juizo de Agua Fria por fallecimento do seu pae e sogro Bernardo da Silva. De modo que si o alferes José da Silva e Oliveira não fosse filho de Bernardo,seria genro. Nada de certo é sabido sobre a descendencia do alferes José da Silva e Oliveira,ou mesmo de Francisco da Silva e Bernardo da Silva, que os dois genealogistas a que ja me referi dão como filhos de Bernardo da Silva e  que devem ser seus netos, sendo filhos do alferes José da Silva.

Em 1840 foi inventariado na freguezia de Serrinha, termo da  villa de S. João de Agua Fria, comarca de Inhambupe, um José da Silva e Oliveira, deixando os seguintes filhos: Bete Silva e Oliveira, Ludovina Maria do Espirito Santo, Leonor da Silva e Oliveira, que se casou com Antonio Gonçalves Pereira, João Ferreira da Silva, que foi o inventariante (todos maiores), Carlota e Clementina (menores). Morava na fazenda Patos é era viuvo desde 1831, de Anna Rita de Jesús, irmã de Antonio Pedro da Silva e filha de Antonio Ferreira, fallecido proprietario da fazenda Patos.

Esse José da Silva e Oliveira será o mesmo alferes a quem vimos nos referindo? Não. Este em 1763, oitenta e sete annos antes, já era chefe de familia. Era filho? É possivel; é crivel.

Em 1851, Francisco da Silva e Oliveira, Carlos Antonio de Oliveira, Miguel Archanjo de Oliveira, Felippe de Santiago Oliveira, Joaquim de Oliveira e silva, Antonio José de Oliveira, Anna Benedita de S. Bento e Ludovina Maria de, Jesús, casada com Patricio Francisco dos Santos, partilharam amigavelmentes entre si, os bens deixados por sua mãe Anna Maria de Jesús, fallecida em 27 de Outubro de 1849, entre elles as terras fazenda Tiririca; terras em Serrinha, o descoberto do Sipó nas terras da Tiririca,o descoberto da Cajazeira em terras da Tiririca.

Francisco da Silva e Oliveira, Xiquinho da Tiririca enviuvou em 10 de Abril de 1854 de sua mulher Anna Custodia de Jesús, que lhe deixou cinco filhos - Anna, José, Maria, Francisco e Bernarda. Anna, o mais velho dos filhos, tinha apenas 14 anos. Pergunta-se: quem era o pae de Francisco da Silva e  Oliveira? A mãe era, como já vimos, Anna Maria de Jesus, que também se dizia Anna Maria Joaquina.

Tinha tios - Pedro, Joanna, Ludovina e tambem José Coitinho. Seu pae deve ser algum filho do alferes José da Silva e Oliveira. Não é possivel-que seja este, porque à sua mãe é Anna Maria de Jesús, ou Anna Maria Joaquina, e a mulher do alferes José da Silva e Oliveira chamava-se Anna de Jesus e Silva; e, depois, a sua geração não é a dos netos, mas a dos bisnetos de Bernardo da Silva, pelo tempo que viveu, em 1854.

Ha tempos morreu em Serrinha Francisco da Silva e Oliveira, conhecido por Nonô da Tiririca. É filho deste Francisco da Sil
eira e de sua mulher Anna Custodia, esta filha de Seraphim Seraphim de Oliveira Maya, nasceu em 1846, figurou no titulo de herdeiros do inventario de sua mãe semente com o nome de Francisco e tinha irmãos e tios.

Será possivel que os seus parentes não saibam esclarecer este assumpto? Será possivel que não tenham documentos velhos? Ainda é
tempo de apurar-se o passado deste ramo da familia de Serrinha. É só não se descuidarem. O que com algum esforço se pode fazer Hoje, talvez daqui a alguns annos seja de todo impossivel.

O que não padece duvida é que Francisco da Silva e Oliveira não é filho, nem irmão de José da Silva e Oliveira inventariado em 1840, porque não figura como filho no inventário deste, e no de sua mulher que foi feito em 1831, nem no inventario da mãe de Francisco figuram os filhos de José da Silva, como sobrinhos representantes do pae premorto. Dest'arte, si Francisco da Silva
for filho do tenente José da Silva e Oiveira, não sel-o-á José da Silva e Oliveira, inventariado em 1840 e, si este o for, não sel-o-á Francisco da Silva e Oliveira, que então deverá ser filho de algum filho do tenente José da Silva e Oliveira, talvez um Francisco da Silva, ou um Bernardo da Silva, mais provavelmente um Francisco da Silva, ou mesmo Francisco da Silva e Oliveira,
que então será o terceiro desse nome e o mais velho de todos, a quem o genealogistas da familia, professor Martins e padre Cupertino, dão como filhos do velho Bernardo da Silva, um dos quaes com e outro sem descendencia, e que em verdade serão seus netos.


