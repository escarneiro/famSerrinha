\chapter{Os Silvas}
\begin{centering}
	\section*{Familia do Genipapo}
\end{centering}
Ignacio Manoel da Silva, figura na \textit{Genealogia da Familia de Serrinha}, do professor Antonio Martins, como genro de Bernardo da Silva. Mas na do padre Cupertino, que se diz orientado por informações da familia, elle é apontado como filho.
Comquanto nenhum documento tenha para resolver as duas opniões, sigo esta ultima, por ser de um tetraneto, que cuidadosamente investigou o caso.

Teve Ignacio Manoel da Silva tres filhos, a saber, O capitão Antonio Manoel da Silva que casou com Ignacia Maria de Lima, foi
procurador a Casa da Ponte e firmou-se na fazenda Pedra,que comprou a esta Casa; d. Thereza de Jesús Maria, morta pouco antes de 1846, que se casou com Ignacio Gonçalves Pereira e morou no Genipapo; Felicidade, que não casou.

O capitão Antonio Manoel da Silva teve 8 filhos: João Manoel da Silva, que se casou com uma exposta; Antonio Manoel da Silva, que sempre se conservou celibatario; José Manoel da Silva, aleijado, que também se conservou celibatario; Anna Maria da Silva, que nunca se casou, Maria, que se casou com Manoel Hilario de Araujo e foi morar em N, S. da Conceição do Coité; Antonia, que se casou com Affonso da Silva Cardoso, filho de Affonso Martins da Silva e sua mulher Maurícia Ribeiro da Silva, natural da villa de N. S. do Coração de Jesús do Pedrão, de onde se mudou em 29 de Janeiro de 1821 para Serrinha, casando-se em 10 de Outubro do mesmo anno e fixando-se na fazenda Vargem Velha; Ignez, que se casou com José Alexandre de Araujo; Rosa Maria de Lima, que se casou, em 21 de Outubro de 1821 com Francisco Joaquim de Araujo, seu primo e irmão de José Alexandre de Araujo, e morreu em 2 de Abril de 1871, com 65 annos de idade.

A prole de Thereza foi inferior a de seu irmão capitão Antonio Manoel apenas em uma unidade, Efectivamente teve ella sete filhos: Geronymo Gonçalves Pereira, que se casou e retirou-se para Oliveira dos Campinhos, onde foi residir; José Ramos de Oliveira, que se casou em primeiras nupcias com Maria Francisca de Oliveira, filha de Antonio Manoel da Motta, familia do Tambuatá, viuva duas vezes, e em segundas nupcias com Maria Ramos, família da Serra Grande, e morreu em 26 de Outubro de 1847; Vicente Ferreira da Silva, que se casou com Bernarda da Silva Pinheiro, filha do capitão Manoel de Affonseca Pinheiro, ramo dos Affonsos, morta em 1878; José Pereira Pinto, que foi casado com Antonia da Silva Pinheiro, filha do capitão Manoel de Affonseca Pinheiro e irmã da mulher de Vicente Ferreira da Silva e morou no Morro das Pombas; Joaquim Manoel da Silva, Juca, que se casou com Maria das Mercês; filha de Custodio Ferreira Passos e Rosara do Calando; Maria da Penha, que se se casou com Manoel, Ferreira de Oliveira, flho Manoel Ferreira Santhiago, ramo dos Santhiagos, Serra Grande, e Anna, tia Naninha, que não se casou sobre quem, diziam os antigos, em suas preces, vinha sempre pousar a pomba de Divino Espirito Santo.

Não conhecemos todos os tetranetos de Bernardo da Silva, por via de seu neto, capitão Antonio Manoel da Silva, da Pedra.

Sabemos apenas que João Manoel da Silva, casado com uma exposta, teve sete filhos, dos quaes uma flha se casou com Placido José Ferreira e outra com Ricardo; Ferreira de Oliveira.

Maria, casada com Manoel Hilario de Araujo, do Coité, teve os filhos seguintes: \textbf{I}. João Paulo de Araujo que se casou com Maria Francisca de Jesús, a muda, filha-de Manoel Joaquim de Oliveira e Mafia Francisca de Jesús, ramos dos Motta e dos Santhiagos, morta em 1851, deixando os seguintes filhos Manoel Francisco da Silva, que depois passou a chamar-se Mandel Pablo de Araujo, Anna Firmina da Silva, que se casou com Antonio Manoel de Oliveira, Pedro Fancisco da Silva, que morreu na adolescencia, e Josepha Alexandrina da Silva, e, enviuvando; casou-se com Maria Cardoza da Silva, morta em 1860, de cujo consorcio teve Maria Ritta, Victoriano, Manoel, José Germano e Francisca; \textbf{II}. Joaquim Rufino de Araujo, que se casou com Maria Josepha, filha de Custodio Francisco Junqueira, portuguez, e Antonia, filha de Apollinario Ferreira, gente da Serra Grande, e se estabeleceu em S. Caetano; \textbf{III}. José Lopes da Silva, casado que foi com Maria do, Carmo, filha de João Pinheiro Alves de Souza; \textbf{IV}. Manoel Hilario de Araujo, que se-casou com Maria Mariinha,: filha de Manoel Ferreira da Silva, e se firmou. no Morro Redondo; \textbf{V}. Lino, tio Lino; \textbf{VI}. Candido, que não se casou; \textbf{VII}. Rosalina Maria de Jesús, que se casou, em primeiras nupcias, com José de Oliveira Santhiago, morto em 1850, de quem teve José Izidro de Oliveira, Malachias José de Oliveira, Joaquim José de Oliveira, Manoel José de Oliveira, José, Maria Rosalina de Jesús, Anna, Josepha, Jacintha, Quintiliano e Maria (posthema), e em segundas nupcias com Placido José Ferreira; \textbf{VIII}. Senhorinha, que se casou com Pedro dos Patos, filho de Antonio Pedro da Silva.

Antonia e seu marido Affonso da Silva Cardozo foram paes de Antonio Manoel da Silva, Dé, que não se casou; João Paes Cardozo, que foi casado com Jesuina (filha de Vicente Ferreira da Silva); Manoel Cardoso da Silva, Bidinho, que se casou com Anna Maria (filha de José Alexandre de Araujo); Maria Maximiana de Jesús que foi casada. com Gordiano de Souza Estrella; Anna das Brotas que se casou com José Alexandre de Araujo, viuvo de Ignez sua tia; Francisca Clementina do Amor Divino, que foi casada com Manoel Joaquim de Araujo, viuvo de Maria Alexandrina; Ritta Maria de Jesús, que se casou com Geraldo da Silva Mendonça, Siruga; Maria Ritta de S. Joaquim e Antonia Francisca de Jesús.

Os filhos de Ignez e seu marido José Alexandre de Araujo foram: Manoel Longuinho de Araujo, que se casou no Coité com Joanná Cyrylla, irmã do coronel João Manoel Amancio e por lá ficou, tendo tido muitos filhos; Leoncio, Manoel, José, João, Joaquim e Antonio. Maria Alexandrina casada com seu primo Manoel Joaquim do Nascimento; Constança casada com seu primo João Manoel de Araujo, irmão de Manoel Joaquim do Nascimento, Anna Maria José, casada com Manoel da Silva Cardoso, e Antonia Maria de Jesús, casada com Manoel Vicente de Souza, morta em 1886, deixando os filhos seguintes: Maria Amalia Cardoso casada com José Pedro Cardoso, Antonio Marques de Souza, Eduviges Leonor de Souza, Jose Olympio de Souza e Maria: moravam ma Chapada.

Rosa Maria de Lima, que se casou em 21 de Outubro de 1821 com Francisco Joaquim de Araujo, o velho Titi, e morreu em 2 de Abril de 1871 com 65 annos de idade, teve os seguintes filhos: Maria Carolina de Lima, nascida em 28 de Fevereiro de 1823 e morta em 13 de Dezembro de 1904 não se casou; Alexandrina Maria de Lima, nascida em 28 de Março de 1825,casou-se com Joaquim Ferreira de Oliveira, Pimpim, de cujo consorcio teve Antonio, José(coronel José Ferreira de A. Oliveira, Ducas), Maria, José Pedro, Joaquim, Manoel Joaquim e Miguel, morou sempre na fazenda Vargem Velha e morreu em 5 de Junho de 1868; Manoel Joaquim do Nascimento, nascido em 25 de Dezembro de 1828, casado em primeiras nupcias com Maria Alexandrina, morta em 18 de Dezembro-de 1854, e em segundas nupcias com Francisca, filha de Affonso da Silva Cardoso, e morto em 13 de Março de 1886; João Manoel de Araújo, nascido em 6 de Maio de 1829, casado com Constança, filha de seu tio José Alexandre de Araújo, e fallecido em 6 de Agosto de 1900; capitão José Joaquim de Araujo, meu pae, nascido em 1º de Maio de 1831, casou-se em primeiras nupcias, em 18 de Setembro de 1856, com Antonia Clementina Moreira Pinto, minha mãe, filha do portuguez Manoel José Pinto e sua mulher Bernarda Archangela Moreira, fallecida em 28 de Maio de 1871, deixando desse seu consorcio filhos, a saber, padre José de Cupertino e Araújo Lima, vigario de S. Anna do Catú, Cecilia, (Dadate), que não se casou, Maria Herminia de Araujo Ribeiro, Sinhá Lia, viuva do major Symphronio Cardoso Ribeiro, Reginaldo, que morreu de febre amarella em 1877 no collegio Athenêo Bahiano, da Capital, do qual era alumno, Laudelina Candida de Araujo, Sinhá Dona, solteira, eu, bacharel Antonio José de Araujo, Juiz de Direito da Comarca de Jacobina, e Joaquim José de Araujo, fallecido de febre amarella em 11 de Janeiro de 1886 em Serrinha, onde gosava as ferias collegiaes, alumno que era do Collegio Victoria, na Capital; casou-se em segundas nupcias, em 13 de Dezembro de 1871, com Carolina Carneiro, filha do capitão José Carneiro da Silva e sua mulher Geronyma, Loló, e fallecida em 28 de Agosto de 1909, deixando desse. seu consorcio estes filhos: padre José Alfredo de Araujo; fallecido como vigario de Alagoinhas, professoras Josephina, casada com Afro Freitas, e Etelvina, casada com Rosalvo Mendonça, dr. Salvador Reginaldo de Araujo, juiz municipal do Catú, João Ferreira de: Araujo, industrial em Ilhéos, Carolina, Alzira, Amelia e Elisa, casada com
Antonio Freitas, negociante abastado em Beritingas; chefiou o partido conservador em Serrinha de 1868 a 1889, quando se proclamou a Republiça. e abandonou a politica, e morreu, como um justo, em 24 de Dezembro de 1909, na villa do Catú, cercado dos cuidados dos filhos e magoado pelas saudades de todos que o conheceram, bonissimo que era: a elle, aos seus esforços, labor e prestigio, deve a Serrinha o ter sido elevada a villa em 1876, na presidencia do dr. Henrique Pereira de Licena, mais tarde Barão de Licena, o politico mais sincero, mais leal, mais dedicado aos amigos, mais reconhecido e prestante que já tive a fortuna de conhecer e com quem tive a honra de manter relações muito estreitas por apresentação que de mim lhe fez meu pae em 1885, quando me matriculei na Faculdade de Direito do Recife; Anna Senhorinha de Lima, nascida a 1º de Abril de 1833 e fallecida em 25 de Janeiro de 1913; Redogina Maria de Lima, nascida à 13 de Março de 1835 e morta em 9 de Dezembro de 1917
casada que foi com Manoel Francisco de Oliveira; e, finalmente, Antonio Joaquim de Oliveira; nascido em 9 de Maio de 1839, casado com Anna, filha de Joaquim José de Oliveira e Joanna Pimentel, e fallecido em 9 de Julho de 1886.

Pelo lado de Thereza, casada que foi com Ignacio Gonçalves Pereira, os tetranetos de Bernardo da Silva foram: os filhos de Geronymo Gonçalves Pereira dos quaes não tenho noticia; os tilhos de Vicente Ferreira da Silva, cujos nomes já declinei no capitulo relativo aos Affonsos, a cujo ramo pertencia a mulher de Vicente; os filhos de José Ramos de Oliveira; morto em 26 de Outubro de 1847, com sua primeira mulher, Maria Francisca, filha de Antonio Manoel da Motta( Tambuatá), os quaes serão declinados no capitulo relativo aos Mottas, familia do Tambuatá, nenhum tendo havido com sua segunda mulher, Maria Ramos; os filhos de José Pereira Pinto, a saber, Anna, que se casou com Joaquim Bento de Souza, Virgilia, que se casou com José Antonio da Silva, Maria Leopoldina, que se casou com o major João Pereira das Mercês, Cecilia, que não se casou, Geronyma, que se casou com Joaquim Cordeiro, paes de Basilio Cordeiro de Almeida, e Anna Rozenda, que se casou com. Francisco Cordeiro, paes de José Cordeiro de Almeida, os filhos de Maria da Penha, casada que foi com Manoel Ferreira de Oliveira, dos quaes fallaremos no
capitulo destinado aos Santhiagos, familia da Serra Grande.

Joaquim Manoel da Silva, Jaca, que se casou com Maria das Mercês, não deixou descendencia. %the end

