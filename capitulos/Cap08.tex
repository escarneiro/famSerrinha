\chapter{Os Mottas}
\section*{}
Escreve Quaresma, no seu roteiro: «Do tanque (refere-se ao tanque de Serrinha) a Tambuatá ha uma legua e um quarto e aqui
tem seus moradores e cria-se gado e está na ponta de uma serra e mais adiante fica um caminho a parte direita que vae dar em uma nascente de agua.

Daqui pode ir mais adiante tres quartos de legua, a Massaranduba, que é sobre um alto de uma serra e tem um morador e agua abaixo da casa para a parte do sudoeste e pastos não faltam.» Isto escreveu elle em 1731.

Nesse tempo, pois, já o Tambuatá tinha morador. Tinha-o mesmo antes, em 1716. Bernardo da Silva ahi morava então.

Em 1785, sessenta e nove annos depois da primeira noticia da moradia de Bernardo, 1716, e cincoenta e quatro após a passagem de Quaresma, Tambuatá pertencia a Francisco Manoel da Motta. Isto consta de uma escriptura publica de accordo, ou composição, e venda, que em 2 de Dezembro desse anno, «neste sítio da Serrinha e Igreja della, freguezia da villa de Sam Joam de Agua Fria, termo della», em notas do tabellião Luiz Antonio Vieira, passaram Fructuoso de Oliveira Maya e sua mulher Bernarda Maria da Silva, como vendedores, a José Maria Rodrigues, como comprador, desistindo aquelles da demanda que com este tinham em Agua Fria sobre o sítio Campinas, que elles haviam comprado ao illustrissimo Manoel de Saldanha, e vendendo-o a-este por 270\$800, a prazo.

Pelas divisas dadas nessa escriptura a esse sitio, parte elle, «pela parte do poente, pela malhada das Aboboras e procurará o rumo ao alto da Alagôa da Pedra e Alagôa da lenha do Gato thé o rio Subahé e dahi subirá, pelo dito rio acima com suas voltas e enseadas, thé a Tabôa e dahi correrá o rumo pela dita Tabôa acima thé topa: com terras do Tambuatá, de Francisco Manoel da Motta, ficando esta confrontando pelo dito rumo do meio que fizer com o dito Tambuatá e Campinas, e da dita malhada das Aboboras correrá rumo direito para a parte do Norte thé topar com terras da Bocca da Catinga e por esta parte se demarcará com os mais heréos suas extremas, rumos e confrontações» e «com o Subahé se demarcará pelo riacho chamado a Grota funda e dahi correrá o rumo pelo dito rumo thé finalisar o dito rio e dahi largando o rio correrá o rumo a buscar a malhada alta thé topar com o rumo que parte as Campinas com o Tambuatá.»

José Ferreira Santhiago foi fiador de José Maria Rodrigues e Antonio Ferreira Santhiago, o capitão Apollinario da Silva e o sargento mór Manoel de Jesus Gouveia foram testemunhas instrumentarias da escriptura. Francisco Manoel da Motta é, como se vê della, proprietario do Tambuatá, que, já existindo em 1716,devia ter passado por mais de um proprietario, depois que os antepassados da Casa da Ponte, então ainda por nascer, o venderam. Francisco Manoel da Motta era neto de Antonio Manoel
da Motta e de sua mulher, filha de Bernardo da Silva, proprietarios, antes de Francisco Manoel da Motta, da fazenda Tambuatá, naturalmente por herança de Bernardo da Silva e sua mulher Maria do Sacramento.

Antonio Manoel da Motta teve os seguintes filhos: Antonio Manoel da Motta, que se casou com Anna Maria, sua prima, filha de Domingos Ferreira Santhiago e sua mulher Antonia Maria, da Serra Grande; uma filha casada com Francisco Moncorvo, de Cachoeira; Emerenciana, casada com Apollinario Ferreira, fundador da fazenda Lagedo, ramo dos Santhiagos; uma filha casada com José Ferreira de Oliveira, de Dois Irmãos; Josepha, casada com Antonio Ferreira de Oliveira Santhiago, o licenciado, e foi morar em Massaranduba.


Antonio Manoel da Motta, o filho. teve os filhos seguintes: Padre Antonio Manoel de Oliveira, do Retiro: onde em 1829 era capellão; Francisco Manoel da Motta, do Tambuatá, que se casou com Josepha; José Manoel da Motta, que se casou com Anna Francisca, Caldeirão, Retiro; Anna Maria de Oliveira minha bisavó, que se casou com José da Cunha Araujo, de Coité, meu bisavô, e se estabeleceu no Sobradinho; Maria Francisca de Jesus, inventariada em 1839, foi casada em primeiras nupcias com Manoel Joaquim de Oliveira, em segundas nupcias com Antonio Joaquim de Oliveira e em terceiras nupcias com José Ramos de Oliveira, morto em 26 de Outubro de 1847; Maria Josepha, que se casou em primeiras nupcias com Custodio Francisco Joaquim, portuguez, e em segundas nupcias com Antonio Gonçalves de Araujo, de S. Caetano; uma filha casada com Antonio Ferreira Santhiago, filho do licenciado, e outra casada com Vicente Ferreira Ramos, gente dos Santhiagos, que, enviuvando della, passou a segunda nupcias com Maria Francisca, natural do Coité, morta em 1867,com quem teve muitos filhos.

Emerenciana, casada com Apollinario. Ferreira, teve muitos filhos, dos, quaes fallarei quando tratar dss Santhiagos (familia de Serra Grande), a cujo ramo pertence Apollinario.

A filha casada com José Ferreira de Oliveira, de Dois Irmaos, teve sete filhos, de que tratarei up capitulo relativo aos Santhiagos, familia de Serra Grande, à qual pertence seu marido.

Josepha, casada com Antonio Ferreira Santhiago, o licenciado, teve tres filhos, dos quaes me occuparei no capitulo destinado aos Santhiagos, familia da Serra Grande, Santhiago que é o seu marido.

Isto posto, vejamos, por via dos Mottas, os tetranetos de Bernardo da Silva e sua: mulher: 

Tetranetos de Bernardo da Silva, bisnetos da filha casada com Antonio Manoel da Motta, netos de Antonio Manoel da Motta Filho e filhos de Francisco Manoel da Motta, do Tambuatá: Seraphim Manoel da Motta, casado em primeiras nupcias com Maria Paim da Silva, filha de Antonio de Mattos Paim (ramo dos Affonsos) e inventariada em 1843, deixando os seguintes filhos: Fernando, Anna, Joanna e Maria, e terras no Marruaz, e em segundas nupcias com Ludovina Coitinho; Francisco, Chiquinho do Retiro, que se casou com Benedicta e morreu no mesmo dia em que ella morreu; Maria Francisca, que se casou com Angelo José de Oliveira, morto em 22 de Maio de 1850, em estado de viuvez, tendo do consorcio Maria Jesuina, Anna, que se casou com José Carneiro de Oliveira José, Francisco e Virginia, que veio a se casar com Antonio Joaquim de Araujo; Francisca, que se casou com José Martins Valverde, viuvo de Maria Francisca da Silva, morta em 1842, como melhor se verá no capitulo relativo aos Santhiagos, a cuja familia pertencia elle; Antonio Manoel da Motta, Sirioli.

Tetranetos de Bernardo da Silva, bisnetos de Antonio Manoel da Motta, o pae, netos de Antonio Manoel da Motta, o filho, e filhos de José Manoel da Motta: Zézinho, do Saquinho; Chiquinho; Vicente, do Caldeirão; Silveria e Anna, casada com José Marcellino.

Tetranetos de Bernardo da Silva, bisnetos da filha casada com Antonio Manoel da Motta, o velho, netos de Antonio Manoel da Motta, o moço, e filhos de Anna Maria de Oliveira casada com José da Cunha Araujo: José Alexandre de Araujo, que se casou em primeiras nupcias com Ignez, filha do capitão Antonio Manoel da Silva, ramo dos Silvas, e em segundas nupcias com Anna, filha de Affonso. Martins da Silva e neta do capitão Antonio Manoel da Silva, portanto sua sobrinha affim; Francisco Joaquim de Araujo, que morreu em 25 de dezembro de 1891 com 93 annos, e foi casado com Rosa Maria de Lima, filha do capitão Antonio Monoel da Silva; José Antonio, que se casou com Dona, irmã de Luiz Gonçalves, Luizinho, da Tabúa; Manoel José da Cruz; Joaquim
de Araujo; Maria Francisca de Araujo, que se casou com Luiz Gonçalves, Luizinho, e Anna, Naninha, que se casou com Antonio Joaquim de Almeida, foi morar em Monte Alegre e lá ficou, tendo tido tres filhas, a saber, Maria Innocencia casada com Francisco Ferreira da Silva, Anna Francisca casada com José Justiniano de Lima Branco, piloto, e Antonia casada com Francisco José de Araujo, e muitos netos e bisnetos.

Tetranetos de Bernardo da Silva, bisnetos da filha casada com Antonio Manoel da Motta, o velho, netos de Antonio Manoel da Motta, o moço, e filhos de Maria Francisca de Jesus: Antonio Manoel da Motta e José de Oliveira, filhos do seu primeiro marido, Manoel Joaquim de Oliveira, Maria, que se casou com João Paulo de Araujo, filha do seu segundo marido, Antonio Joaquim de Oliveira, e Maria Francisca, que se casou com José Martins Valverde, Angelo Ramos de Oliveira, mudo, João Nepomuceno Ramos, mudo, e Innocencio Ramos de Oliveira, filhos de seu terceiro marido, José Ramos de Oliveira.

Tetranetos de Bernardo da Silva, filhos de Antonio Manoel da Motta, o velho, netos de António Manoel da Motta, o moço, e filhos de Maria Josepha: Custodio, que, se casou com uma filha de José Alves Ferreira, Cajueiro; Francisco Brasileiro, que se casou com Custodia, filha de José Alves Ferreira, Cajueiro; e Maria, que se casou com Joaquim Rufino de Araujo; todos filhos do seu primeiro marido Custodio Francisco Junqueira.

Tetranetos de Bernardo da Silva, bisnetos de Antonio Manoel da Motta, o velho, netos de Antonio Manoel da Motta, o moço, e filhos de Antonio Ferreira Santhiago, filho do licenciado, e de Vicente Ferreira Ramos: encontrar-se-ão no capitulo destinado aos Santhiagos, onde devem ser procurados.

Nada sei da descendencia da filha de Antonio Manoel da Motta, casada com Francisco Moncorvo, de Cachoeira. Della descende o dr. Tiberio Moncorvo Lima, que foi presidente da Província da Bahia.

O que ahi fica, entretanto, e sufficiente para facilitar a reconstituição da familia nas lacunas que existem e não puderam ser evitadas, diante da escassez de recursos de que dispuz.
