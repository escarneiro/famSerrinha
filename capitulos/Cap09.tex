\chapter{Os Santhiagos}
\section*{}
Dos filhos, homens e mulheres, de Bernardo da Silva o que lhe deu maior descendencia foi Antonia Maria da Silva. Casando-se com
Domingos Ferreira Santhiago, natural de Iguape, foi morar na fazenda Serra Grande, que, como já vimos, em 1716 foi comprada, com as Fazendas Dois Irmãos e Massaranduba a d. Isabel Maria Guedes de Britto, viuva do coronel Antonio da Silva Pimentel, pelo capitão Antonio Homem da Affonseca Correia.

Tendo sido esta fazenda partilhada entre seus herdeiros, naturalmente então lhe pertencia.

Domingos Ferreira Santhiago teve do seu consorcio com Antonia Maria da Silva os seguintes filhos: José Ferreira de Oliveira, que se casou com uma filha de Antonio Manoel da Motta, do Tambuatá; e se estabeleceu no sitio Dois Irmãos; Joaquim Ferreira Santhiago, que se casou com uma senhora da fazenda Trindade, e se collocou no sitio Brejo; Manoel Ferreira Santhiago, que se casou primeiro com uma senhora da fazenda Picada, no arraial de Pedrão, e depois com Maria da Conceição, da fazenda Cajueiro, em Uriçanga; Antonio Ferreira Santhiago, o licenciado, que se casou com Josepha, filha de Antonio Manoel da Motta, do Tambuatá, e estabeleceu-se na fazenda Massaranduba; Apollinario Ferreira, que se casou com Emerenciana, filha de Antonio Manoel da Motta, do Tambuatá, e se firmou na fazenda Lagedo; Anna do Rosario, que não se casou e residiu no sitio S. Anna; Josepha que se casou com João Pinheiro Alves de Soaza, irmão da mulher do capitão Apollinario da Silva; Bernardina, que se casou com Antonio Gonçalves Pereira, e foi morar no Saquinho; Cypriana que se casou, não sei o nome do seu marido, creio que é Pedro da Silva, e fundou a fazenda Candeal; Anna Maria, que se casou com Antonio Manoel da Motta, filho do Tambuatá, e fundou a fazenda Retiro e duas filhas, cujos nomes ignoro, casadas respectivamente com um tal Costa e com Matheus da Silva Cardozo, que tambem foi marido de Ignacia, filha de Antonio Carneiro da Silva.


Filhos-de José Ferreira: de Oliveira: Antonio Ferreira de Oliveira, que se casou com Anna Joaquina de Jesus, filha de João Pinheiro Alves de Souza, e foi morar no sitio Vargem da Baixa; José Ferreira de Oliveira, morto em 1846, casou-se com Anna Francisca da Silva, filho de Matheus da Silva Cardozo, e residiu na fazenda Dois Irmãos; Francisco Ferreira da ressurreição, que se casou com Maria do Carmo, filha de João Pinheiro Alves de Souza, e morou em Dois Irmãos; Manoel Joaquim de Oliveira, casado com Maria Francisca, fllha de Antonio Manoel da Motta, do Tambuatá, o moço, que delle enviuvou; Antonio Joaquim de Oliveira, que se casou com sua cunhada Maria Francisca, viuva de seu irmão Manoel Joaquim; Ignacia casada com José Pinheiro Alves de Souza, de Dois Irmãos, morta antes de 1860, Anna Maria, que se não casou. 

Filhos de Joaquim Ferreira Santhiago: Antonio Alves Ferreira Santhiago, que e casou com uma filha de José Gonçalves Pereira e não teve filhos; José Ferreira Santhiago, casado com uma senhora da povoação de Queringola, e estabelecido na fazenda do Tanque Novo, segundo o professor Antonio Martins; José Alves Santhiago, meu Alves, que se casou não sei com quem; Francisco, Xico Bufa, que se casou com Maria Thomasia, filha de Manoel José Pirajá; Maria Josepha, que se casou com um Gonçalves Pereira, Domingos ou Antonio; Ignacia que se casou com José Alves de Queiroz.

Filhos de Manoel Ferreira Santhiago: Vicente Ferreira Ramos, que se casou em primeiras nupcias com uma filha de Antonio Manoel da Motta, do Tambuatá, e em segundas nupcias com Maria Francisca de Jesus do Coité, morta em 1867 (ao seu primeiro casamento com uma senhora da fazenda Picada, no Pedrão); Manoel Ferreira de Oliveira, que se casou com Maria da Penha, filha de Ignacio Gonçalves Pereira, e se estabeleceu na fazenda Genipapo; José Alves Ferreira, que se casou com Maria Joaquina, filha de Antonio Ferreira Santhiago, filho do licenciado, e collocou-se em Cajueiro; Zacharias Ferreira da Silva e Oliveira, que se casou com Anna, filha de Custodio Ferreira Passos, do Calando; José Ferreira de Carvalho, que nasceu em 1783 e morreu em 1866, casou-se com Maria Rosaria de Lima, filha de Antonio Martins da Silva, natural do Pedrão e irmã do padre José Alves, e foi morar em Campo Limpo; Antonio Alves de Carvalho, que se casou com Felicidade, irmã de Antonio Manoel, da Feira de S. Anna, e se firmou na Fazenda Estiva, freguezia de Bom Despacho; Antonio Ferreira de Oliveira, que se casou com Ritta da Cunha, da Queimada do Curral; Thereza, que se casou com Joaquim Gonçalves Pereira: Maria da Corôa, que se casou com Antonio Ferreira Santhiago e, enviuvando, contrahiu casamento com Luiz Lopes da Silva (do seu segundo casamento com Maria da Conceição, que por escriptura particular de 28 de Julho de 1823 vendeu a seu primo Affonso da Silva Cardoso dez mil reis dos noventa que tinha nas terras de Serra Grande, meiação no inventario do seu marido, assignando por ella seu filho José Ferreira de Carvalho e sendo testemunhas José Martins Ferreira, Antonio Ferreira do Nascimento e Zacharias Ferreira da Silva Oliveira).

Filhos de Antonio Ferreira Santhiago, o licenciado: Antonio Ferreira Santhiago, que se casou primeiro com uma filha de Antonio Manoel da Motta, o moço, do Tambuatá e depois, enviuvando, com Maria da Corôa, filha de Manoel Ferreira Santhiago; João Apollinario Ferreira, que se casou com Joanna, filha de Apollinario Ferreira, e ficou no Cantinho;
Acna Maria, que não se casou; Maria Joaquina, que foi casada com José Alves Pinheiro.

Filhos de Apollinario Ferreira: Antonio Ferreira, Antonio cantador que se casou com Francisca, filha de José Nazario da Costas; Antonia que se casou com Custodio Francisco Junqueira, portuguez; Theodora e Marianna, que não casaram; Josepha, que se casou com José Nazario; uma filha que se casou com Leonardo, portuguez, e teve uma filha conhecida por Finca; Joanna, casada com João Apollinario; Maria Josepha, Consolo, casada com Antonio Gonçalves Ferreira, Gangorra; uma filha casada com José Vicente da Costa.

Filhos de Josepha casada com João Pinheiro Alves de Souza: José Pinheiro Alves de Souza, que se casou com Ignacia, filha de José
Ferreira, de Dois Irmãos; José Alves Pinheiro, casado com uma senhora de Agua Fria; Anna Joaquina, Anninha da Vargem, casada com Antonio Ferreira de Oliveira, filho de José Ferreira de Dois Irmãos; Josepha, que se casoucom o capitão Joaquim Ferreira Baptista, de Inhambupe; Maria do Carmo, que foi casada em primeiras nupcias com Francisco Ferreira da Resurreição e em segundas com José Lopes da Silva: Felicidade, que se casou com Antonio de Carvalho, portuguez, morador no logar Ilha, freguezia de Santa Barbara.

Filhos de Bernardina casada com Antonio Gonçalves Pereira, do Saquinho: Antonio Goncalves Pereira, do Saquinho: Antonio Gonçalves Pereira, Gangorra, que se casou, primeira vez; com Maria Josepha, Consolo, filha de Apollinario Ferreira, segunda vez com uma filha de Manoel José Pirajá e, terceira vez; com Antonia, filha de José Alves Ferreira, da fazenda Brejo; Domingos, casado com Maria Josepha, filha de Joaquim Ferreira Santhiago, do Brejo; José Gonçalves Pereira, casado com Maria Sant'Anna, filha de Matheus da Silva Cardoso em suas nupcias com uma filha de Domingos Ferreira Santhiago e Antonia Maria, da Serra Grande; Joaquim Pereira Gonçalves, morto em 1846, casádo que foi com Maria Vicencia de Oliveira, fallecida antes de 1846, filha de José Ferreira de Oliveira, do Socavão, de cujo consorcio teve Maria, casada com João Paulo de Araujo, segunda mulher, Anna casada com José Ferreira de Oliveira Gomes, Maria Francisca, Maria Clementina, Antonia, Bernarda, Joaquim, Jesuina e Ludovina; Francisco Goncalves, casado com Maria Finca, filha de Custodio Francisco Junqueira; José de Oliveira e  Manoel Gonçalves, que coustituiram familias, mas das quaes nada sabemos eu e o padre Cupertino, meu irmão, de cujas notas me tenho servido neste capitulo, e até mesmo o professor Martins, pertencente aliás este ramo da familia de Serrinha; Anna, que se casou com Antonio Joaquim Gonçalves, do Coité, irmão de Luiz Gonçalves, Luizinho; e Francisca, que se casou com Prudente Manoel da Silva, do Candeal. 

Filhos de Cypriana casada, creio, que com Pedro da Silva, Candeal: Prudente Manoel da Silva, que se casou com Francisca, filha de Antonio Gonçalves Pereira, e José Felix, que não se casou.

Filhos de Anna Maria, casada com Antonio Manoel da Motta, o moço, do Tambuatá: do capitulo relativo aos Mottas, familia do Tambuatá.

Filhos da filha casada com um tal Costa: Nazario José da Costa, que não se casou e se estabeleceu na cidade de Cachoeira; José Nazario que se casou com Josepha, filha de Apollinario Ferreira, Barro; José Vicente da Costa, que tambem se casou com uma filha de Apollinario Ferreira, Sucupira; e Anna Maria, que se casou com Manoel José Pirajá, Sitio do Meio.

Filhos da filha casada com Matheus da Silva Cardozo: foram seis, diz o padre Cupertino, mas não dá o nome senão de Josepha, casada com F. Gomes, Maria S. Anna casada com José Gonçalves Pereira e Anna casada com José Ferreira de Oliveira, de Dois Irmãos. São todos bisnetos de Bernardo da Silva.

Vejamos agora os seus tetranetos por via de Domingos Ferreira Santhiago e Antonia Maria.

Joaquim Ferreira de Oliveira, Pimpim, casado com Alexandrina Maria de Lima, filha de Francisco Joaquim de Araujo e Rosa Maria de Lima, morta em 1869, em Vargem Velha; Antonio Ferreira de Oliveira, casado em primeiras nupcias com Maria Joaquina, filha de Francisco Ferreira da Ressurreição, e em segundas nupcias com uma filha de Placido José Ferreira, fazenda Pombal; Miguel Ferreira de Oliveira, inventariado em 1868, casado que foi com Senhorinha Constança de Oliveira, de cujo consorcio teve Anna, Maria, Antonio e Gaspar (Gaspar Ferreira de Oliveira); Tertuliano Ferreira de Oliveira, Lino, casado com  Rosenda, filha de José Martins Valverde do Carrapato; Maria Francisca de Oliveira, casada com Joaquim Cárneiro da Silva Ribeiro, filho do capitão José Carneiro da Silva; Anna Francisca de São João, que se casou com José Pedro Coititinho, Tiririca; Claudina Ferreira de Oliveira, Senhorinha, que se Casou com José Ferreira Cannabrasil (todos filhos de Antonio Ferreira de Oliveira, inventariado em 1865, e netos de José Ferreira de Oliveira); José Martins Valverde, que se casou em primeiras nupcias com Maria Francisca da Silva, filha de José Ramos de Oliveira e Maria Francisca de Jesus, familia Motta, Tambuatá, morta em 1842, com quem teve dois filhos, Anna e José Martins Valverde, e em segundas nupcias com Francisca, filha de Francisco Manoel da Motta, do Tambuatá, e em terceiras nupcias com uma filha do capitão João Gomes de Carvalho; Maria Vicencia, que se casou com Joaquim Pereira Gonçalves, do Socavão, morto em 1846 (filhos de José Ferreira de Oliveira e netos de José Ferreira de Oliveira, pae, de Dois Irmãos); Ricardo Ferreira de Oliveira, morto em 1864, que foi casado com Francisca Maria de Jesus, que lhe sobreviveu, filha de José Manoel da Silva, do Boi Manso, de cujo consorcio deixou os seguintes filhos: José Thomé de Oliveira, Manoel Barnabé, Anna Maria de Jesus, Maria, Rosenda, Marcolina, Antonio e Josepha; Maria Joaquina, que se casou com Antonio Ferreira de Oliveira (filhos de Francisco Ferreira da Ressurreição e netos de José Ferreira de Oliveira, de Dois Irmãos); Antonio Manoel da Motta, casado com Francisca Rosa de Lima, filha de José Ferreira de Carvalho, do Carrapato; José de Oliveira, casado com Rosalina, filha de Manoel Hilario de Araujo, ramo dos Silvas (filhos de Manoel Joaquim de Oliveira casado com Maria Francisca de Jesus, filha de Antonio Manoel da Motta, do Tambuatá, e netos de José Ferreira de Oliveira); Maria, muda, que se casou com João Paulo de Araujo (filha de Antonio Joaquim de Oliveira e sua mulher Maria Francisca de Jesus, viuva do seu irmão Manoel Joaquim, e neta de José Ferreira de Oliveira), João Ferreira de Oliveira, casado em primeiras nupcias com Maria de Jesus, filha de Antonio Ferreira Santhiago, e em segundas com Anna, filha de José Gonçalves Pereira, do Brejo; Antonia, que se casou com Antonio Ferreira da Motta; Vicente Ramos, que tambem se dizia Vicente Ferreira de Araujo, que se casou com Maria Alexandrina da Silva, inventariada em 1866, com cinco filhos, Manoel, Maria, Anna, Antonia e Constança; Antonio Manoel da Motta, que apparece ás vezes com o nome de Antonio Manoel de Araujo, em documentos velhos; Marianna Maria de Jesus, morta antes de 1867. casada que foi com Joaquim Pinheiro de Carvalho, de cujo: consorcio teve Maria, casada com Miguel Alves Santhiago, Manoel Pinheiro, Antonio, José, João e Joaquim; Maria da Porciumcula de Oliveira, casada com Florencio Terreira da Silva; Maria Ritta de Jesus casada com José Longuinho da Cunha (netos de Manoel Ferreira Santhiago e filhos de Vicente Ferreira Ramos, os dois primeiros com sua primeira mulher, filha de Francisco Manoel da Motta, e os demais com sua segunda mulher Maria Francisca, do Coité); Joaquim Manoel de Oliveira, que se não casou; Sulpicio Ferreira de Oliveira, inventariado em 1875, que foi casado com Rita Carneiro, filha de Matheus Carneiro da Silva, e teve os seguintes filhos: Sulpicio Trancolino de Oliveira,
Joaquim Carneiro de Oliveira, Candido Carneiro de Oliveira, que morreu em menoridade, e Anna Ritta Carneiro de Oliveira, que foi casada com Joaquim Quintino de Oliveira e com elle teve quatro filhos, a saber: Maria, Isaias, Anna e Idalina; Geronyma, Loló, que se caso, primeiras nupcias, com Zacharias Gonçalves Pereira e, segundas nupcias com o capitão José Carneiro da Silva, Zuza do Mandacarú, pae de Meu Zé, da finada Calú, segunda mulher do capitão Zezinho, meu pae, etc.; Justina, que se casou com Ludovico Antunes de Carvalho, primeira mulher; Anna, que se casou com o tenente coronel Joaquim Carneiro de Campos, primeira mulher; Joanna, Janjana, que foi casada com o capitão Antonio Cardoso Ribeiro (filhos de Manoel Ferreira de Oliveira e sua mulher Maria da Penha, e netos de Manoel Ferreira Santhiago); Manoel Ferreira Santhiago, casou-se com uma filha de José Alves de Queiroz; Joaquim Ferreira Santhiago, casou-se com uma filha de Zacharias Ferreira da Silva Oliveira; Anna, casou-se com Manoel Alves Campos; Maria Eleuteria, casou-se com Francisco Brasileiro; Antonia, casou-se com Custodio e depois com Antonio Queiroz, Gangorra; Rosalina, casou-se com Francisco Alves (Pedo); Maria Rosa, casou-se com Antonio Manoel; Jo- %autor se passou nesse pedaço aqui, parece.

casou-se com José Ferreira da Silva (filhos de José Alves Ferreira e Maria Joaquina e netos de Manoel Ferreira Santhiago); Joaquim Ferreira da Silva, casado com uma filha de Joaquim Alves de Queiroz; José Ferreira da Silva, casado com uma filha de José Alves Ferreira (Santo Christo), Capucho; Zacharias Ferreira da Silva, casado com Theodora Maria de Jesús, filha de Vicente Ferreira Ramos, Campo Redondo: Maria Messias, casada com Angelo Ferreira de Carvalho; Francisca Rosa, casada com Luiz Lopes Ferreira da Silva; Rosaria, casada com Joaquim Ferreira Santhiago e uma moça casada com Pedro Lopes da Silva (filhos de Zacharias Ferreira da Silva Oliveira e netos de Manoel Ferreira Santhiago): Angelo Ferreira de Carvalho, casado em primeiras nupcias com Anna Bernardina Moreira, filha de Francisco Manoel Amancio da Cunha, ramo dos Mayas, e em segundas nupcias com Maria Messias, filha de Zacharias Ferreira da Silva Oliveira; professor Antonio Martins Ferreira, autor da Genealegia da Familia de Serrinha, obra inedita, escripta ha mais de cincoenta annos, consideravelmente additada por meo irmão padre Cupertino, de cujos subsidios tenho me utilisado bastante neste trabalho, o qual se casou em primeiras nupcias com Maria, filha do capitão Manoel de Affonseca Pinheiro, não tendo filhos, e ém segundas nupcias com Anna Francisca, filha do capitão José Carneiro da Silva, com quem teve cinco filhos, do que fallamos no capitulo dos Affonsos; Severo Fabiano de Carvalho, que se casou com Maria Moreira, do Raso, ramo dos Mayas; Ludovico Antunes de Carvalho, que se casou com Justina, filha de Manoel Ferreira de Oliveira e, enviuvando, com Rita, filha de Pedro Alves Pinheiro; Francisca Rosa de Lima, tia Ló, que secasou com Antonio Manoel da Motta, Pedra; Antonia, que se casou com Angelo Pastor Ferreira; Carlota, interdicta; Maria Fidelis casada com José Thomé Ferreira (filhos de José Ferreira de Carvalho e netos de Manoel Ferreira Santhiago); Virginio Ferreira de Oliveira, casou-se com Ritta Constantina, filha de José Ferreira de Carvalho, Madeiras; Angelo Pastor Ferreira, que se casou com Antonia, filha de José Ferreira de Carvalho, Camamum; João Ferreira de Oliveira, casado com Maria, filha de Antonio Manoel da Motta, Calderão; Anna, que se casou com Hygino Ferreira da Motta; Antonia, casada com Satyro Lopes Guimarães (filhos de Antonio Ferreira de Oliveira e sua mulher Ritta da Cunha e netos de Manoel Ferreira Santhiago e sua segunda mulher Maria da Conceição); Zacharias Gonçalves Pereira, casado que foi com Jeronyma, Loló, filha de Manoel Ferreira de Oliveira, Genipapo; Maria da Conceição, que se casou com Antonio Alves Pinheiro (filhos de Thereza e seu marido Joaquim Elísio Pereira e netos de Manoel Ferreira Santhiago); Luiz Lopes Ferreira da Silva, casado com Francisca Rosa, filha de Zacharias Ferreira da Silva Oliveira, estabelecido na Retirada; Vicente Lopes de Araujo, casado com uma filha de Claudio Lopes, Cantinho; Pedro Lopes da Silva casado com uma filha de Zacharias Ferreira da Silva Oliveira, Cantinho; Francisco Lopes da Silva, casado com uma filha de Claudio Lopes, Poço (filhos de Luiz Lopes da Silva e sua mulher Maria da Corôa, que com seu primeiro marido, Antonio Ferreira Santhiago, teve um lilho José Thomé Ferreira, e netos de Manoel Ferreira Santhiago); Antonio Ferreira da Motta, casado com Antonia, filha de Vicente Ferreira Ramos, Licory; Maria de Jesus. casada com João Ferreira de Oliveira; José Thomé Ferreira, casado com Maria Fidelis, filha de José Ferreira de Carvalho, Raso (o ultimo filho de Antonio Ferreira Santhiago, filho do licenciado do mesmo nome, com sua Segunda mulher Maria da Corôa, e os dois primeiros com sua primeira mulher, filha de Antonio Manoel da Motta. do Tambuatá, e todos netos de Antonio Ferreira Santhiago, o licenciado); João que se casou com uma filha de Antonia, do Pé da Serra, Cantinho; e Honorata, que se casou com Matheus, filho de José Gonçalves Pereira (filhos de João Apollinario Ferreira e netos de Antonio Ferreira Santhiago, o licenciado); os netos de Antonio Ferreira Santhiago, o licenciado, filhos de Maria Joaquina, e seu marido José Alves Ferreira, vão nomeados com os netos de Manoel Ferreira Santhiago, pae deste; José Francisco Junqueira, morto em 1841, casado que foi com Anna Christina de Jesus, filha de José Pinheiro Alves de Souza, de Dois Irmãos, com quem teve uma filha Maria, que em 1841 tinha doze annos; Antonio Francisco Junqueira, Mimim, que foi casado com Victoria, filha de José da Silva, Lagêdo; Maria, Finca, que se casou com Francisco Gonçalves Pereira, (filhos de Antonia casada com Custodio Francisco Junqueira, e netos de Apollinario Ferreira); Antonio Manoel de Oliveira, casado com Maria Rosa, filha de José Alves Ferreira, Brejo; Josepha,que se casou com José Ferreira, Brejo; Francisca, que se casou com Antonio Ferreira, Brejo (filhos de Josepha casada com José Nazario e netos de Apollinario Ferreira); os filhos de Joanna, casada com João Apollinario Ferreira, já foram declinados; Anna, que se casou com Antonio Gonçalves, do Coité (filha de José Vicente e neta de Apollinario Ferreira): Pedra, Alves Pinheiro, morto em 1876, casado com Izabel Carolina de Souza, filha do alferes Coelho, e deste consorcio teve estes filhos: Ritta Alves Pinheiro casada com Ludovico Antunes de Carvalho, viuvo, Antonia Pinheiro de Souza casada com Antonio Alves Pinheiro, Anna Carolina de Souza casada com Miguel Antunes de Oliveira, João Alves Pinheiro; Antonio Alves Pinheiro e Pedro Alves Pinheiro Filho; alferes Antonio Alves Pinheiro, casou-se com Maria da Conceição, filha de Joaquim Gonçalves Pereira, Mucambo; Anna Christina, morta antes de 1860, casou-se com José Maria Ferreira da Motta, com quem teve um filho, Virginio; Maria Ramos de Jesus, casou-se com José Ramos de Oliveira, filho de Thereza, ramo dos Silvas, e viuvo de Maria Francisca, família do Tambuatá, e não teve filhos (filhos de José Pinheiro Alves de Souza e Ignacia e netos de Josepha e seu marido João Pinheiro Alves de Souza); já descrevemos os filhos de Anna Joaquina e seu marido Antonio Ferreira de Oliveira e netos João Pinheiro e Josepha; nada sabemos dos filhos de José Alves Pinheiro, netos de João Pinheiro, coronel Ezequiel, Antonio Joaquim e uma senhora, residentes no Termo da Villa do Itapicurú filhos de Josepha casada com Joaquim Baptista, de Inhambupe, netos de João Pinheiro Alves de Souza; os filhos de Maria do Carmo e seus maridos, primeiro e segundo, vão na casa dos seus paes; Onofre, Miguel, Maria e uma outra moça (filhos de Felicidade e seu marido Antonio Carvalho, Ilha, S. Barbara, netos de João Pinheiro Alves de Souza); Antonio, o gago; Maria, casada com. Joaquim Affonso José Gonçalves Pereira (filhos de Antonio Gonçalves Pereira, Gangorra e netos de Antonio Gonçalves Pereira e Bernarda); José Gonçalves Pereira, casado com Maria Eleuteria de Jesus, filha de José Alves Pereira, Inchú; Maria; um interdicto (filhos de Domingos ou Joaquim Gonçalves Pereira e netos de Antonio Gonçalves Pereira); Matheus, casado com Honorata, filha de João Apollinario Ferreira; Anna, casada com João Ferreira de Oliveira; Maria, casada com Antonio Ferreira Santhiago (filhos de José Gonçalves Pereira e netos de Antonio Gonçalves Pereira e Bernarda); Antonio Manoel de Oliveira, casado com Maria Rosa, flha de José Alves Ferreira, Barro; Francisca, que se casou com Antonio Ferreira; Josepha, que se casou com José Ferreira (filhos de Jose Nazario e netos de Costa); Anna, que se casou com Antonio Gonçalves, do Coité (filha de Jose Vicente e neta de Costa); José Quintiliano, que se casou com Magdalena, filha de Antonio Lopes; José da Costa, que não se casou; Maria Thomazia, que se casou com Francisco Santhiago; Margarida, que se casou com Vicente de Andrade, Tamarindo; mais tres moças, das quaes duas se casaram respectivamente com Francisco da Salgada e Antonio Gonçalves Pereira e outra não se casou (filhos de Anna Maria casada com Manoel José Pirajá).

Os tetranetos de Bernardo da Silva, por via de sua neta Anna Maria, encontram-se no capitulo dos Mottas, à cuja familia pertence seu marido Antonio Manoel da Motta.

A lista dos tetranetos de Bernardo, por sua filha Antonia Maria da Silva, já vae longa; ainda assim é incompleta. Completem-n'a os seus descendentes, bastante que, para isto, é o cabedal que ahi fica, colhido com muitos  esforços, bastante paciencia e sobretudo tenacidade.


