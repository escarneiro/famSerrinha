\chapter{Os Carneiros}
\section*{Familia de S. Bartholomêo}

Em 12 de Janeiro de 1741, por escriptura publica passada na cidade do Salvador, Bahia de Todos os Santos, Antonio Carneiro da Silva comprou à Santa Casa de Misericordia, representada por seu procurador Francisco de Sá Peixoto, a fazenda S. Bartholomêo, no Termo da Villa de N. S. do Rosario do Porto de Cachoeira.

Vinte e cinco annos depois, em 21 de Maio de 1766, Antonio Carneiro da Silva comprou a Manoel de Saldanha, por escriptura publica passada em notas do tabellião da cidade do Salvador, Bahia de Todos os Santos, Antonio Barbosa de Oliveira, do sítio Bom Successo, Boqueirão e Tocós, «que parte ao meio com a fasenda da Serrinha, pelo Morro».

E quarenta e tres annos mais tarde, em 2 de Janeiro de 1784, por escripturas publicas em notas do Tabellão Antonio Pinto dos Reis, de são João de Agua Fria, no sitio de Serrinha, o mesmo Antonio Carneiro da Silva e sua mulher Anna Maria da Silva ratificaram os dotes, que haviam feito a seus filhos Ignacio Manoel Carneiro, Maria de Jesus de Assumpção casada com Braz Ferreira da Costa,. alferes José da Silva Carneiro, Antonia Maria da Silva. Ignacio teve a fazenda Bom Successo, Braz a fazenda Boqueirão, o alferes José da Silva a fazenda Tanque, Antonia a fazenda Poções e Anna a fazenda Cedro.

Não conheço outras escripturas de ratificação de dotes. Mas, alem destes filhos, Antonio Carneiro da Silva e sua mulher Anna Maria tiveram ainda os seguintes filhos: João Baptista Carneiro, que se casou com uma senhora da familia Rego, do Pedrão, e se estabeleceu em São José de Itapororocas; Antonio da Silva Carneiro, morador em sua fazenda São  Bartholomêo, que por escriptura publica de 22 de Setembro de 1814, no arraial de Serrinha, em casa de Maroel José Moreira, genro de Fructuoso de Oliveira Maya, comprou á Condessa da Ponte, por seu procurador capitão Antonio Manoel da Silva, a fazenda Vermelhos; Rosaria, que se casou com Custodio Pereira Passos, portuguez, e se estabeleceu no sitio Calando; Ignacia que se casou com Matheus da Silva Cardoso, Petêo, estabeleceu-se no sitio Cajazeira; uma filha que se casou em S. José de Itapororocas e cujo marido fundou a fazenda Agua Fria.

Antonio da Silva Carneiro só teve dois filhos: Antonio Carneiro, de Agua Dôce, que não teve posteridade, e o capitão José Nunes da Silva Carneiro, que em 1825 vivia em sua fazenda S. Bartholomêo, termo de Cachoeira, e nesse anno, em 25 de Novembro, tomou de em-


prestimo ao cofre de orfãos de Jacobina a quantia de 400\$000 e deu em garantia as suas fazendas Calderãozinho e Vargem das Flores, com gado, sendo seus abonadores Domingos Ferreira Fidelis e Joaquim José de Sant'Anna e sua mulher Antonia Maria da Luz: era casado com Maria de Jesus das Mercês e foi representado por seu procurador Manoel Cypriano de Souza.

João Baptista Carneiro teve os seguintes filhos: coronel José Baptista Carneiro, que em 7 de Janeiro de 1811, por escriptura publica passada na villa e minas de S, Antonio de Jacobina, onde se casou com Anna Simplicia de S. Catharina, constituiu diversos procuradores, entre elles José Carneiro da Silva Rego, em Cachoeira, e João Carneiro da Silva Rego, em qualquer
parte, para defendel-o n'uma questão que lhe moveu Christovam Pereira da Silva, cirurgião-mór de um dos regimentos da cidade da Bahia; José Carneiro da Silva Rego, morador no Termo de Cachoeira; João Carneiro da Silva Rego,que em 1819 era tabellião na Bahia; Bernarda que não casou e se firmou no sitio Malhada de Areia.

O Alferes José da Silva Carneiro. que morreu em 28 de Janeiro de 1857 com a idade de noventa e oito annos e alguns mezes, casou em primeiras nupcias com uma sua prima, filha de Miguel Affonso Ribeiro, e em segundas nupcias com sua prima segunda Anna Maria, filha do	capitão Manoel José Moreira.


Do seu primeiro casal teve dois filhos: o capitão José Carneiro da Silva, que se casou com Maria Francisca da Purificação, filha de José Affonso e neta de Miguel Affonso Ribeiro, e Anna da Silva Carneiro, que se casou com Tristão Gomes da Silva, neto de Fructuoso de Oliveira Maya. Do seu segundo casal houve os seguintes filhos: Bernarda, que não se casou; Joaquim Moreira da Silva Carneiro, que casou com Joanna de Lima, do Pedrão, e se estabeleceu no sítio Malhada e teve muitos filhos; Ricardo Moreira da Silva Carneiro, que foi casado com Catharina Mathildes Carneiro, filha do coronel José Baptista Carneiro, sua prima, foi morar em Riachão de Jacuhype e não teve filhos; Maria da Representação, casada que foi com Manoel José Vieira, seu parente, ramo dos Apollinarios; Luiza Prudenciana, que se consorciou com Antonio Alves Carneiro, seu primo; Anna Ritta, que se casou com Francisco Simplício, filho do coronel José Baptista Carneiro.

Rosaria casada com Custodio Ferreira Passsos, portuguez, é mãe do capitão José Ferreira da Silva, que se collocou no sitio Victoria, de Antonio Ferreira da Silva, que se firmou no sitio Aboboreiras, de Joaquim Ferreira da Silva, de Custodio, de Anna Maria da Silva, que se casou com Zacharias Ferreira da Silva Oliveira, seu parente, ramo dos Santhiagos, e Maria das Mercês, que foi casada com Joaquim Manoel da Silva.

Os filhos de Ignacia, casada com Matheus da Silva Cardoso, Petêo, foram: Matheus Carneiro da Silva, que se casou com Anna Francisca, sua prima, filha de José Affonso; Anna Maria, que se casou com José Affonso Ribeiro, seu primo, e Bernarda Maria da Silva, que se casou com seu primo Miguel Affonso Ribeiro.

Nada sei sobre a descendencia dos outros filhos de Antonio Carneiro da Silva e sua mulher Anna Maria da Silva.

O coronel José Baptista Carneiro, seu neto, filho de João Baptista Carneiro, que em 1811 morava em Jacobina, onde se casou com Anna Simplicia de S, Catharina, das familias Miranda e Brandão, e foi o fundador da fazenda Agostinho Duarte, distante duas leguas de Bom Despacho, municipio da Feira de Sant'Anna, teve de seu casal os seguintes filhos, todos tetranetos de Bernardo da Silva: Maria, casada com José Ferreira da Silva, Zuzinha, de Aboboreiras, pae do Dr. João Marcellino da Silva Carneiro, já fallecido; Francisco Simplício da Silva Carneiro, casado com Anna Ritta, filha do alferes José da Silva Carneiro; João, casado na familia Moreira Rego, do Iassú; José Baptista casado na mesma familia; Antonio Alves Carneiro. casado com Luiza Prudenciana, filha do alferes José da Silva Carneiro, de cujo consorcio teve cinco filhos, a saber, Ricardo, Juvencio, Maria da Pureza, conego José Carneiro, vigario do Pedrão, e Antonio; Guilhermina, casada com Antonio Ferreira da Silva, da Aboboreira; Theodora, casada com Bernardino Ferreira, de Aboboreiras; Joaquim Baptista Carneiro, casado na familia Ayres de Almeida, do município de Sant'Amaro; Luiza, solteira; Anna, casada com José Tavares, Zuzú, de S. Ritta; Catharina Mathildes Carneiro, casada com Ricardo Moreira da Silva Carneiro, da Lagôa do Boi, filho do alferes José da Silva Carneiro; Candida, casada com Bernardo Carneiro, da fazenda S. Antonio, de cujo consorcio nasceu o coronel João Paulo da Silva Carneiro;Miguel Alves Carneiro, casado, sem filhos; José Carneiro da Silva Rego, tambem seu neto, que foi tabellião na Bahia, teve os seguintes filhos: bacharel José Carneiro da Silva Rego, que se
casou com uma senhora pernambucana e foi figura saliente na Sabinada; José Rego, que não se casou; uma filha casada com o tenente Marinho; uma outra casada com José Correia de Britto.
Si outros filhos teve, como é possivel, não é do meu conhecimento.

Dos filhos do alferes José da Silva Carneiro só conhecemos a descendencia do capitão José Carneiro da Silva, de Anna da Silva Carneiro casada com Tristão Gomes da Silva, de Maria da Representação casada com Manoel José Vieira e de Luiza Prudenciana casada com Autonio Alves Carneiro. Eil-a: 

Filhos do capitão José Carneiro da Silva, casado que foi com Maria Francisca da Purificação, filha de José Affonso e neta de Miguel Affonso Ribeiro: \textbf{I} Joaquim Carneiro da Silva Ribeiro, casado com Maria Francisca de Oliveira, filha de Antonio Ferreira de Oliveira, ramo dos Santhiagos, o qual se estabeleceu em Agua Bôa,e teve varios filhos; \textbf{II} Capitão José Carneiro da Silva, Zuza, morador em sua Fazenda Mandacarú, casado com Jeronyma, Loló, filha de Manoel Ferreira de Oliveira, ramo dos Santhiagos, e viuva de Zacharias Gonçalves Pereira, de cujo consorcio teve os seguintes filhos: Carolina Carneiro de Araujo, Calú, casada com o capitão José Joaquim de Araujo, capitão Zezinho, meu pae, viuvo de Antonia Clementina Moreira Pinto Jose Carneiro da Silva, meu Ze, anda vivo e quasi octogenario, Antonio, Elpídio e Leonel; \textbf{III} Tenente-coronel Miguel Carneiro da Silva Ribeiro, pae de Antonio, Joaquim e Marianno Silvio Ribeiro, e de Theodolina, Dulina, casada com Pedro Alves Pinheiro, Miquelina casada com Antonio Rodrigues Nogueira, Anna, Donana, e Leopoldo; \textbf{IV} Capitão Ignacio Carneiro da Silva Ribeiro; \textbf{V} Capitão Tertuliano da Silva Ribeiro, que se casou com uma senhora de Inhambupe; \textbf{VI} Antonio Carneiro da Silva Ribeiro; \textbf{VII} Elpidio Carneiro da Silva Ribeiro; \textbf{VIII} Anna Francisca Carneiro, casada com o professor Antonio Martins Ferreira, ramo dos Santhiagos, viuvo de Maria Pinheiro, ramo dos Affonsos, com quem teve os seguintes filhos: padre Loreto, padre Urbano, professora Amalia (Vina), Pacifica e uma filha que se casou com o negociante portuguez Santos Seara, residente na Capital, hoje representada, bem como seu marido Já fallecido, por seus filhos João Martins dos Santos Seara e Maria Izabel Martins Seara, residentes no Rio de Janeiro; \textbf{IX} Maria Rosa Carneiro casada que foi com o professor Manoel, Cardoso Ribeiro.

Filhos de Anna, casada com Tristão Gomes da Silva: José Bento, que se não casou. Não tenho noticia de outros.

Filhos de Maria da Representação, Sinhá do Sacco, casada com Manoel José Vieira, ramo dos Apollinarios: Liberato que morreu na adolescencia, e Manoel José Vieira, que se casou com Anna Maria, filha de Joaquim José Vieira e Anna
Cardoso, e deixou dois filhos — Alfredo e Manoel.

A descendencia de Ignacia, casada que foi com Matheus da Silva Cardoso, Petêo, por não ter se afastado de Serrinha é toda conhecida. São seus netos, filhos de seu filho Matheus Carneiro da Silva, inventariado em 1864: \textbf{I}Josê Pedro, que não contrahiu casamento; \textbf{II} Manoel Cardoso Ribeiro, que se casou com Maria Rosa, filha do capitão José Carneiro da Silva; \textbf{III} Tenente-coronel Joaquim Carneiro de Campos, que se casou em primeiras nupcias com Anna, filha de Manoel Ferreira de Oliveira, e em segundas nupcias com Cesaria, filha do Tenente João Manoel de Freitas, tendo filhos de um e outro casal, entre elles Campos Filho, que morreu como director da Secretaria de Polícia da Bahia, Maria casada com Leovegildo Cardoso Ribeiro e Carolino Carneiro de Campos, todos do primeiro casal; \textbf{IV} Alferes Jesuino Carneiro da Silva, que se casou com Luiza, filha de Manoel José Pinto, portuguez, ramo dos Mayas; \textbf{V} Capitão Antonio Cardoso Ribeiro, que se casou com Joanna, Janjana, filha de Manoel Ferreira de Oliveira, ramo dos Santhiagos, e teve os seguintes filhos: Tiburtina casada com Leoncio Marques Pedreira de Freitas, Symphronio, Tiberio, Leovegildo, Joaquim, Antonio, Miguel e Augusto Cardoso Ribeiro e uma filha de nome Maria e appellido Moça, viuva, estes dois ultimos ainda vivos; \textbf{VI} Alferes Rosendo Carneiro da Silva, casado que foi com Santinha, filha de Miguel Affonso Ribeiro, o neto, e sua mulher Bernarda;\textbf{VII} Anna Cardoso Ribeiro, que se não casou; \textbf{VIII} Maria Lina, que se casou com Simão Férreira de Oliveira, filho de Antonio Ferreira de
Oliveira; \textbf{IX} Candida, que se não casou; \textbf{X} Custodia, que se casou com o dr. Benedicto Augusto Wencesláo da Silva; \textbf{XI} Rita, que se casou com Sulpicio Ferreira de Oliveira, filho de Manoel Ferreira de Oliveira; textbf{XII, XIII, XIV} Ignez, Ignacia e Carlota, que se não casaram; \textbf{XV} Carolina, que se casou com Francisco Romão de Araujo, Xico Trovoada.

Quando tratamos dos Affonsos, demos a relação dos filhos de Anna Maria e Bernarda Maria, casadas respectivamente com José Affonso Miguel Affonso Ribeiro, o neto.


A filha de Antonio Carneiro da Silva, casada com o fundador da fazenda Santa Rita, teve uma filha, Anna Maria do Nascimento, que
se casou com Francisco José da Silva, de Maragogipe, que por ahi andava mascateando,morto em 22 de Setembro de 1862, a quem sobreviveu, e teve os seguintes filhos: Francisco Tavares da Silva Carneiro, tenente-coronel José Tavares da Silva Carneiro, Zuzú Tavares, Demetrio da Silva Carneiro, capitão Antonio Tavares da Silva Carneiro e quatro moças, das quaes tres morreram solteiras e uma foi a primeira mulher do capitão José Ribeiro Lima, menino criado por Francisco José da Silva, e morreu ainda jovem, sem descendencia.

Francisco Tavares da Silva Carneiro teve os seguintes filhos: Alexaudrina, casada com Cidronio; a senhora de João José dos Campos; Eduviges, casada com. Quintiliano Martins da Silva, Feira de Sant'Anna; Maria Angelica, tenente-coronel Francisco Tavares, José Tavares, Sidô, Manoel Tavares, Joaquim. Tavares, Jaco, Martiniano Tavares, todos da Silva Carneiro; Maria Magdalena, casada com Tiberio Constancio Pereira; Francisca, casada com João Ferreira, Badé, e Antonia, casada com Manoel Ferreira, Badé.

O tenente-coronel José Tavares da Silva Carneiro teve os seguintes filhos; capitão Carlos, capitão Joviniano e Jovina, casada com o capitão Leopoldo Ferreira da Silva Carneiro.

Demetrio da Silva Carneiro, casado com Francisca Maria da Silva, filha do capitão José Ferreira da Silva, teve os filhos seguintes: capitão José Carneiro da Silva Ferreira, morador em Itapeipú, Jacobina; Leopoldo Ferreira, Demétrio Ferreira, ambos da Silva Carneiro; Maria Magdalena, casada com o capitão Carlos, seu primo, Antonia Leopoldina, casada com Aristides Cedraz de Oliveira; Izabel Ferreira, casada com Joaquim Cedraz, irmão de Aristides; Maria, casada com José Cedraz, idem; Luiza, casada com Octaviano Cedraz, idem; Franscisca, casada com Annibal Ferreira; Leopoldina, casada com José Baptista Carneiro; Alexandrina, casada com o tenente Juvencio Alves; Umbelina, casada com Ricardo Alves e Anna, casada com Jayme Moreira.

S. Rita fica entre S. José de Itapororocas, de onde dista uma legua e Tanquinho, no municipio de Feira de S. Anna, e os descendentes do fundador de S. Rita, acima nomeados, são tetranetos de Bernardo da Silva, o fundador de Serrinha.


São tetranetos de Bernardo da Silva, por via de Rosaria casada com Custodio Pereira Passos: O tenente-coronel José Ferreira da Silva, Casusa da Victoria, que nunca se casou; O coronel Manoel Ferreira da Silva, pae dos drs. Quintino e Jacintho Ferreira da Silva e sogro do dr. Manoel Ribeiro Lima; Justino Ferreira da Silva, do Carrapato; João Ferreira da Silva, que morreu solteiro; Francisca Maria da Silva, que se casou com Demetrio da Silva Carneiro; Joanna, que se casou com o capitão Antonio Tavares da Silva Carneiro; Bernardina Maria da Silva, Sinhazinha, que se casou com João Ferreira da Silva; Maria das Mercês, Bembem, segunda mulher do capitão José Ribeiro Lima, do Zumby, pae do dr. Manoel Ribeiro Lima (filhos do capitão José Ferreira da Silva); Antonio, José, João Ferreira e as senhoras do coronel Manoel Ferreira da Silva e de Justino Ferreira da Silva, esta chamada Luiza (filhos de Antonio Ferreira da Silva); Ponciano, João e outros (filhos de Custodio Ferreira;) Manoel Ferreira da
Silva, que se casou com uma senhora da familia Medeiros, de Mata Grande (filhos de Joaquim Ferreira da Silva).

Nada sei quanto aos filhos de: Maria das Mercês, si os teve, e pelo que diz respeito aos de Anna Maria da Silva, casada com Zacharias Ferreira da Silva e Oliveira, delles já fallei, quando escrevi sobre os Santhiagos, a cujo ramo pertence seu pae.



