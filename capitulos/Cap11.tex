\chapter{Os Maias}
\section*{Familia de Serrinha}
Por escriptura publica passada na cidade do Salvador e Bahia de Todos os Santos, em casa de morada do illustrissimo Manoel de Saldanha, pelo tabellião Manoel Antonio Campello, em 17 de Novembro de 1759, o mesmo illustrissimo Manoel de Saldanha e sua mulher dona Joanna da Silva Guedes de Britto, clegitimamente senhores e possuidores de um sítio chamado o Saquinho e o dos Possos nas suas terras do certão dos Tocós», ahi graphado toquôz, «com trez leguas de comprido e duas de largo, o qual confronta por todas as partes com quem de direito fora, venderam-n'o por 600\$000 (seiscentos mil reis) que no acto receberam, a Fructuoso de Oliveira Maya, que nessa escriptura se diz ser «morador nos certões». A venda foi feita com «declaração porem que nesta terra vendida se se achar alguem acituado com arrendamento seu ou de seu procurador, delles vendedores, neste caso ficarão os ditos rendeiros conservados e elle comprador de posse de cobrar delles as rendas não só as vencidas mas tambem as que se forem vencendo ao diante, porem não podem os ditos rendeiros enovar cousa alguma nem  alargarem nem mais daquillo em que estiverem de posse emthé o dia dessa venda e como ha coatro annos desta parte pedindo arrendamento a seu procurador, delles vendedores, Marcos Rodrigues de uma porage a que lhe pozeram por nome Mumbuca deste por ser dentro do dito citio que o comprador occupa e agora compra não tem concedido antes se tem botado abaixo os curraes que o dito Marcos Rodrigues levantou e nesta contingencia tem andado thé agora e neste caso se está dentro das ditas tres leguas de comprido e duas de largo que hé só que vendem ficará este chamado Mumbuca pertencendo ao comprador por não estar ainda acituado o dito Marcos Rodrigues e no caso que o dito logar Mumbuca fique de fóra destas leguas que elles vendedores vendem ficará o dito logar pertencendo a elles ditos vendedores para deste fazerem o que quizerem ou seus procuradores» Foram testemunhas desta escriptura o reverendo vigario Anastacio Pereira e o capitão Luiz de Affonseca Pinto.

Por instrumento publico de auto de posse passado em 6 de Dezembro de 1759, desenove dias depois neste « certão dos Tocós», ahi tambem graphado toquôz, «freguezia e termo da villa de Sam Joam de Agua Fria, em o citio do Saquinho e dos Possos, freguezia e termo da dita villa», O tabellião Manoel Martins Guimarães deu posse judicial desse sitio, parte denominado Saquinho e parte Possos, a Fructuoso de Oliveira Maya, «morador na Serrinha, freguezia e termo da dita villa». Foram testemunhas Manoel Ferreira Santhiago, Bernardo da Silva e Alexandre da Fonseca, tendo este ultimo, por ser analphabeto, assignado de cruz.

Fructuoso era casado cor Bernarda Maria da Silva, filha de Bernardo da Silva. Tiveram oito filhos, a saber: Seraphim de Oliveira Maya, que se casou com Manoela, do Inhambupe, e morreu na casa que depois foi do coronel Miguel Carneiro da Silva Ribeiro e fica no canto da praça da Matriz e da rua Direita, ou Conselheiro Pereira Franco; Bernardina, tia Dina, que se não casou; Anna Maria de Oliveira, casada que foi com o capitão Manoel José Moreira, portuguez; Josepha, que se casou com Tristão de Araujo e Oliveira, portuguez, que se estabeleceu na Mumbuca,; Francisca, casada com Jeronymo, que se estabeleceu nos Possos; Ignacio de Oliveira Maya, que se casou com Anna Maria da Silva, Doninha, sua prima, filha de Antonio Carneiro da Silva; Manoel de Oliveira Maya, que se collocou na Gangorra; e Antonio de Oliveira Maya.

Nas genealogias escriptas pelo professor Martins e pelo padre Cupertino não se faz menção desse Antonio de Oliveira Maya, porem de Antonia de Oliveira Maya, que se casou com José Pereira, portuguez, e foi moradora no Sacco da Matta.

Mas o coronel Aristides Cedraz de Oliveira me afirma não ter tido Fructuoso filha com o nome de Antonia, mas sim filho com o nome de Antonio, e que esse filho é seu bisavô maternó.

Elle ainda me affirma que Fructuoso, seu e meu tetravô, teve realmente oito filhos e que os sítios Saquinho e Possos foram divididos entre elles, por morte de seus paes, e que quatro ficaram com Saquinho e quatro com Possos.

Seraphim de Oliveira Maya teve os seguintes filhos; José da Cruz, que não se casou; Zacharias, que se casou com uma filha de Simão Alves Barretto e Maria Correia, de Agua Fria; uma filha que se casou vom Victoriano Alves Barretto, de Olhos d'Agua; Anna Custodia de Jesus, morta em 10 de Abril de 1854, casada que foi com Francisco da Silva e Oliveira, neto de José da Silva e Oliveira, da Tiririca; e mais quatro filhas, entre ellas Francisca e Pomba, que não se casaram.

Anna Maria de Oliveira, casada com o capitão Manoel José Moreira teve os filhos seguintes: padre José Moreira Maya, que foi capellão de Bento Simão: Manoel José Moreira, que se casou em Cachoeira com uma sobrinha de Manoel Fernandes Serra e se estabeleceu no logar Fojos, em Cachoeira; Anna Maria Moreira, que se casou com seu parente alferes José da Silva Carneiro, viuvo de uma filha de Miguel Affonso Ribeiro, o velho; Maria Moreira, que se casou em primeiras nupcias com José Alexandre e em segundas nupcias com Francisco Manoel Amancio da Cunha, da Bocca da Catinga; Luiza, que se casou com José Lino de Souza e retirou-se para Sentacé (Sento Sé); Bernarda Archangela Moreira, minha avó, que se casou com Manoel José Pinto, portuguez, que em Serrinha se estabelesceu corrido de Cachoeira em consequencia dos movimentos da Independencia e exerceu sempre a profissão de negociante; Antonia Clementina Moreira, que não se casou.

Josepha, casada com Tristão de Araujo e Oliveira, teve estes filhos: Tristão Gomes da Silva, que se casou com Anna, filha do alferes José da Silva Carneiro em suas primeiras nupcias com uma filha de Miguel Affonso Ribeiro, e José Luiz de Araujo, que se casou no Riachão de Jacuhype.

Nada sei de positivo sobre a descendencia de Ignacio de Oliveira Maia, Manoel de Oliveira Maia e Antonia de Oliveira Maia, Em 22 de Maio de 1850 morreu Angelo José de Oliveira, viuvo de Maria Francisca, filha de Francisco Manoel da Motta, de Tambuata. Foi seu inventariante o seu irmão Joaquim de Oliveira Maia, que em 1851 era solteiro, Era irmão de Ignacio de Oliveira Maia. Devem ser tilhos de Ignacio de Oliveira Maia, filho de Fructuoso de Oliveira Maia, E eu creio que sejam, porque eram primos do tenente-coronel Miguel Carneiro da Silva Ribeiro e Ignacio de Oliveira Maia era casado com Doninha, da familia Carneiro, irmã 
do avô do dito coronel Miguel Carneiro, Angelo José de Oliveira deixou os seguintes filhos, que figuraram em seu inventario em 1850 assim: Maria, 18 annos\label{ma18}, Jesuina, 17, Anna, 14, José, 10, Francisco, 8, e Virginia, 6 annos. Maria e Jesuina morreram solteiras; Anna casou-se com José Carneiro de Qliveira e Virginia casou-se com Antonio Joaquim de Araujo, morador na fazenda Recanto, em Serrinha, Tinha terras em Licory, Retiro, patrimonio do finado padre Antonio Manoel de Oliveira, Campinas, Tanque e
Gamelleira. Foi inhumado na freguezia de Riachão de Jacuhype.

O coronel Aristides Cedraz de Oliveira diz me que Manoel Cedraz de Oliveira, seu avô, era irmão de Joaquim de Oliveira Maya e filho de Ignacio de Oliveira Maia; neto, portanto de Fructuoso de Oliveira Maia. Diz-me ainda que seu avô Manoel Cedraz de Oliveira era casado com uma prima, filha de Antonio de Oliveira Maia, de nome Francisca Xista de Oliveira. Acceito a informação, mas propenso a crer, firmado nas informações do professor Martins e do padre Cupertino, que parecem mais seguras, ser sua bisavó que se chamava Antonia de Oliveira Maia e não seu avô que tinha o nome de Antonio de Oliveira Maia, tanto mais quanto elle ignora o nome da mulher do que suppõe ser seu bisavô e os dois informantes a que me refiro dão o nome do marido da sua bisavó, Consigno a divergencia para que, por meio de velhos documentos, que devem existir, os interessados procurem resolvel-a.

O esforço é demasiado para um só, obrigado, na ausencia de archivos publicos, e abandono dos cartorios, procurar papeis antigos em poder de particulares.

Mas si os interessados se entregarem a estas pesquizas, muito podem conseguir, concorrendo para o levantamento da arvore genealogica de suas familias.

Tambem dos tetranetos de Bernardo da Silva, por via de Fructuoso de Oliveira Maia e sua mulher Bernarda Maria da Silva, só conhecemos os que descendem do capitão Manoel José Moreira, casado com Anna Maria de Oliveira. São elles: José Luiz e Maria, filhos de Manoel José Moreira, moradores em Cachoeira, que não se casaram, os filhos de Anna Maria Moreira, cagada com o alferes José da Silva Carneiro, cujos nomes se encontram mo capitulo destinado dos Carneiros, familia de S. Bartholomeu; Jacintho Moreira, que se casou com uma filha de Antonio Pedro da Silva, do Riachão, Manoel Francisco Moreira, que se casou com Constança, filha de Manoel José da Cunha, do Coité, Marianno José da Cunha, que não se casou, Anna Bernardina Moreira, que se casou com o capitão Angelo Ferreira de Carvalho, do Raso (Aracy) e Maria Moreira, que se casou com Severo Fabiano de Carvalho, do Razo (Aracy), irmão de Angelo, todos filhos de Maria Moreira e seu segundo marido Francisco Manoel da Cunha; Manoel Moreira Pinto, Pintinho, que não se casou e morreu velho; José Moreira Pinto, Cazuza Pinto, que tambem não se casou, mas deixou um filho natural, perfilhado, Porphyrio, Anna, que se casou com o tenente José Emygdio Ribeiro, Egydio Moreira Pinto, que se casou com uma senhora de Feira de Sant'Anna, Luiza, que se casou com o alferes Jésuino Carneiro da Silva, Antonia Clementina, minha mãe, que se casou com o capitão José Joaquim de Araujó, capitão Zezinho, meu pae, e Barbara, appellidada de Rainha por sua belleza peregrina, que morreu em pleno noivado, estes filhos de Bernarda Archangela Moreira casada com o portuguez Manoel José Pinto.

Nada sei da descendencia de Luiza casada com José Lino de Souza, de Sentocé (Sento Sé).

Os descendentes de José Luiz de Araujo, filho de Tristão de Araujo e Oliveira, neto de Fructuoso de Oliveira Maia e bisneto de Bernardo da Silva, são: José Luiz de Araujo, filho, Semeão José de Araujo, Adrião José de Araujo, Ignacio José de Araujo e Joaquim José de Araujo, homens, Anna, que se casou com Alexandre, Maria que se casou com Innocencio da Silva Carneiro, Josepha que se casou com Joaquim Justiniano de Oliveira, uma casada com um tio do fallecido coronel Marcolino Gonçalves Mascarenhas, do Riachão, uma casada com Jasé Manoel, filho de Angelo Carneiro, do Gravatá, uma casada com Jósé Cravinho, filho de Angelo Carneiro do Gravatá, uma casada com Innocencio, das Flôres, e Matia José de Oliveira morta em 1922 com setenta e tantos annos de idade, casada com Reinaldo Pereira de Oliveira, natural do Pedrão e morador mo Candeal, de cujo consorcio teve José Temotheo de Oliveira, fazendeiro nó logar Tabúa; Comarca de Jacobina e Antonia Petronilla de Jesus, viuva de Antonio Barnabe Carneiro, filho de Antonio Carneiro da Silva, da Malhadinha e neto de Angelo Carneiro da Silva.

Tristão, imrão de José Luiz de Araujo, teve um filho, José Bento, sique se casou com Maria Simôa, da família Carneiro do Coité, onde fixou residencia, e teve muitos filhos.

Rematando, esclareço que o coronel Aristides Cedraz Oliveira, irmão do faleccido bacharel José Cedraz de Oliveira e outros, é filho de José Cedraz de Oliveira e Maria Joanna, sua mulher, filha de Joaquim Carneiro da Silva, da fazenda Lagôa, terras da velha fazenda Bom Successo, doada a Ignacio Manoel Carneiro por seu pae Antonio Carneiro da Silva, genro de Bernardo da Silva.

José Cedras, pae do coronel Aristides Cedraz, é filho de Manoel Cedraz de Oliveira e sua mulher Francisca Xista de Oliveira e Manoel Cedraz, diz elle ser filho de Ignacio de Oliveira Maia e Francisca  Xista é filha de Antonio ou Antonia de Oliveira Maia, filho ou filha de Fructuoso. São portanto o coronel Aristides Cedraz e seus irmãos bisnetos de Fructuoso de Oliveira Maia. Joaquim da Silva Carneiro pae da mãe do coronel Aristides tinha um irmão chamado Innocencio Carneiro da Silva.
