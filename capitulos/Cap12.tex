\chapter{Emendas e Ementas}
\section*{}
— Supprima-se, na pagina \pageref{surto}, linhas 5 e 6, a oração — O surto munca mais foi detido; e na mesma pagina, linha 15, entre as palavras \textit{Agua-Fria} e as palavras \textit{e as fazendas de,} etc., colloquem-se, entre virgulas, as palavras — villa em 1727.

—Por escripto particular, passado em S. Pedro do Acupe, em 14 de Janeiro de 1749, Manoel de Saldanha e sua mulher Joanna da Silva Guedes de Britto doaram a Heitor da Silva uma legua de terra nas suas terras do sertão dos Tocós chamada o Bom Successo e confrontada e demarcada da maneira seguinte:
buscando para a Vargem, aonde está Miguel Baptista de renda um quarto de legua e para a Queimada Grande outro quarto de legua e para o Riacho dos Angicos trez quartos de legua e para a parte da Bôa Esperança outros tres
quartos de legua que ao todo faz uma legua em quadro não podendo, porem, alienal-o sem primeiro afrontar a elles doadores. A pedido de Antonio Desiderio do Espirito Santo, o tabellião da Bahia, Antonio Barboza de Oliveira, em 12 de Novembro de 1798, tirou uma publica forma desse escripto particular.

As terras a esse tempo eram de Anna Quiteria; que disse em uma petição ao juiz commissario, de Agua-Fria, Dr, Camara, terem-lhe sido doadas pelo illustrissimo Manoel de Saldanha e sua mulher d. Joanna. Nem a petição, nem o despacho do juiz tem data.

— A escriptura publica de venda da fazenda Serrinha a Bernardo da Silva foi passada em 6 de Setembro de 1723, na cidade do Salvador, Bahia de Todos os Santos e pousadas de D. João Mascarenhas, pelo tabellião Manoel Affonso da Costa sendo testemunhas o tenente coronel Manoel de Miranda Oliveira e Luiz Gregório da Cunha. Nella se diz que D. João de Mascarenhas e sua mulher d. Joanna da Silva Guedes de Britto, moradores nesta cidade, vendem a Bernardo da Silva, morador no certão dos Tocós as terras do certão dos Tocós e nellas um sítio chamado a Serrinha, que houveram por
titulo de herança de seu pae e sogro o coronel Antonio da Silva Pimentel, o qual sitio de terra, assim chamado a Serrinha, parte ao meyo por uma parte com terras delles vendedores e sítio em que está de renda Gaspar Pinto, buscando o taboeiro que vae para a catinga aonde morou Antonio Gonçalves demarcando ao meyo com o Saco do Moura donde corre rumo à tapera do Cypriamo de lá cortando a contestar e demarcar com terras de Francisco de Sá Peixoto e da outra parte com o remo a partir com terras do coronel Antonio Homem, pela outra parte corre o dito rumo a partir ao meyo com terras de Manoel Carlos Lima e deste corre rumo direito a contestar e partir com o Saco dos Tapuyas buscando a lagôa chamada do Genipapo correndo o rumo direito a contestar com terras do dito Francisco de Sá Peixoto e assim confrontado e demarcado e com quem mais direitamente deva e haja de partir, confrontar e demarcar com todas as confrontações que direitamente lhe tocam e na forma que foi medido e demarcado no inventario do dito seu paz e sogro e athé hi e depois de sua morte o pessuiram os rendeiros que nelle estiveram de arrendamento com todas as suas entradas e sahidas, logradouros, demais pessuídos, aguas, mattos, estradas reaese particulares, assim em aquella mesma forma que elles outhorgantes o pessuem e melhor si melhor puder ser disseram que muito de sua livre vontade venderam por 2:200\$000, que delles receberam ao passar desta escriptura, Ficam assim rectificados alguns senões que porventura accusem os termos em que, no corpo da obra, esta escriptura foi reproduzida.


— Maria Francisca, moça parda, natural da freguezia de S. Joam Baptista de Agua Fria, filha legitima de Lourenço José da Costa e Anna Joanna de Jesus, estando justa e contractada para casar-se com Manoel Felippe do Espirito Santo, moço de igual qualidade e da mesma freguezia, porem obstando-lhe o impedimento de consaguinidade, em 2º gráo, por ser o pae da contrahente irmão do pae do orador, pediu dispensa, que lhe foi concedida, na Bahia 26 de Agosto de 1829, impondo, entretanto o reverendo senhor vigario «penitencias saudaveis». A informação, favoravel ao pedido, foi dada em Agua-Fria 27 de Julho de 1829 pelo coadjuctor Manoel Paulino Maciel, no governo da freguezia pelo justo impedimento do reverendo parocho. Em Serrinha 8 de Novembro de 1829 o vigario Francisco José de Miranda concedeu licença ao reverendo senhor padre Antonio Manoel de Oliveira, diz elle, «para que em meu logar assista a contracção do Matrimonio dos mesmos na casa de morada de Joaquim de Oliveira, que he decente para levantar Altar portatil; e lhes dê logo a Benção nupcial, impondo-lhes por Penitencias saudaveis o resarem por seis mezes continuos o Terço de N. Senhora; e por obsequio me enviará o assento ao pé desta para se lançar no livro competente.» No mesmo documento o mesmo vigario ainda dá licença ao mesmo padre Antonio Manoel de Oliveira, na mesma data, para « assistir ao consorcio sacramental de José Lourenço da Costa com Anna Vicencia.» Em 8 de Novembro de 1829 o padre Antonio Manoel de Oliveira, do Retiro, era vivo; não o era em 1839, dez annos depois, porquanto messe anno foi inventariada sua irmã Maria Francisca e do acervo do seu casal constam as terras que fóram do patrimonio do Retiro, avaliadas por 112\$220.

— Em Serrinha 28 de Julho de 1823 Maria da Conceição, viuva de Manoel Ferreira Santhiago, vendem por escripto particular, a seu primo Affonso da Silva Cardoso 10\$000 nas terras de Serra Grande, que tinha por meiação no inventario de seu marido e no valor de 92\$000, e bemfeitorias entre as roças do Leonardo e do defuncto Domingos por 13\$000, Por ella assisgnou seu filho. José Ferreira de Carvalho, Foram testemunhas José Martins Ferreira, Antonio Ferreira do Nascimento e Zacharias Ferreira da Silva Oliveira.

— Por escripto particular datado da Fazenda Pedra 10 de Outubro de 1821 Antonio Manoel da Silva e sua mulher Ignacia Maria da Purificação deram de dote á sua filha Antonia Maria por ter casado com seu genro Affonso da Silva Cardoso, uma porção de terras na fazenda Varge, no valor de 50\$000, e mais bemfeitorias no valor de 169\$000. Pela doante assiguou Antonio Manoel da Silva Moço, seu filho.

—Por escripto particular datado da fazenda Pedra 22 de Dezembro de 1822 Antonio Manoel da Silva, (lettra muito boa e do proprio puuho, como a do escripto acima) deu de dote a seu genro Francisco Joaquim, casado com sua filha Rosa Maria, a crionlinha Agostinha, em preço de cem mil reis, o cabrinha Adriano em preço de sessenta e seis mil reis, a mulatinha Francisca em preço de cincoenta mil reis, quatorze oitavas de ouro lavrado e cincoenta mil reis em terras na Fazenda da Pedra.

—Por escripto particular da Fazenda do Tanque 22 de Novembro de 1826 Miguel Affonso Ribeiro e sua mulher Bernarda Maria da Silva venderam a seu primo José Pinheiro de Affonseca uma parte de terras no Saco do Moura que houveram por herança de sua avó Anna da Silva.

— Em petição, dirigida ao juiz de Agua-Fria, Affonso da Silva Cardoso se propôz a justificar que se mudou da freguezia do Santissimo Coração de Jesus de Pedrão, onde nasceu e morava, filho que era de Affonso Martins da Silva e sua mulher Maurícia Ribeiro da Silva, para Serrinha, fazenda Vargem, em 28 de Janeiro de 1821 e devendo ao casal do capitão Manoel da Silva Cunha, então fallecido, seus bens, escravos, animaes, etc., foram tomados violentamente por José Francisco Moreira, morador na Pedra, e seu irmão Antonio José da Cruz.

—As testemunhas da escriptura publica de compra e venda, que em 24 de Outubro de 1763, no sítio da senhora S. Anna de Serrinha, Termo de S. João de Agua-Fria, e casa de morada da senhora Maria do Sacramento, viuva do defuncto Bernardo da Silva, ao reverendo padre Prudente da Silva, Fructuoso de Oliveira Maya e sua mulher Bernarda Maria, o alferes José da Silva e Oliveira e sua mulher d. Anna de Jesus e Silva e Maria da Purificação, passaram a seo irmão e cunhado capitão Apollinario da Silva, de seus quinhões hereditários, por morte de seu pae e sogro Bernardo da Silva, na fazenda Saco do Moura, foram o sargento mór Antonio Telles de Menezes, Florencio Leytam de Mello e Julio Dias Lopes, morador es na villa de Agua Fria.

— Por escriptura publica de 25 de Dezembro de 1834, José Evaristo da Silva e sua mulher Felippa Maria de Jesus venderam a Joaquim José Vieira por 40\$000 a parte de terra que tinham no Sacco do Moura herança de seu pae e sogro Antonio da Silva Braga.

— Interdictada por prodigalidade Josepha da Silva, solteira, filha do capitão Apollinario da Silva, deu-se-lhe por curador a Antonio Alvares Pinheiro. Josepha tinha 73\$500 de terras na fazenda Saco do Moura, herança paterna, e estas terras, postas em praça, foram arrematadas em Agua Fria, em 20 de Janeiro de 1817 pelo sargento-mór Manoel de Jesus da Silva Gomes (ou Gouveia?) a cujas terras eram encostadas as della.

A praça foi presidida pelo juiz triennal Joaquiim José Vieira.

—Por escripto particular de Serrinha, 16 de Novembro de 1844, Zacharias Ferreira da Silva e Oliveira e sua mulher Anna Maria da Silva vendera, a Antonio Manoel da Silva 20\$000 de terras da fazenda do Carrapato, havidas por compra Pela vendedora assignou seu genro Antonio Martins Ferreira, Luiz Lopes da Silva, Francisco Joaquim de Araujo e Vicente Ferreira da Silva, Antonio Manoel da Silva, segundo declarou nas costas desse documento, vendeu 5\$000 dessas terras a seu irmão Joaquim Affonso.

—No inventario de Maria Trancisca, mulher de José Ramos de Oliveira, feito em 1839, foram avaliadores Manoel Affonso Ribeiro e Francisco Barbosa da Silva e foram descriptas e avaliadas as terras do Carrapato (110\$482), as que foram do patrimonio do Retiro (112\$220), as dos Dois Irmãos (99\$570), as do Candeal (85\$500), as do Tambuatá (508000), as da Vargem (11\$000), as de Serrinha (30\$900), as do Sitio (128500), as do Rosario (30\$000) e as da Lagôa do Roi (100\$000).


—No inventario de Manoel José Vieira, fallecido em 9 de Dezembro de 1856, sua viuva, Maria da Representação, deu á carga as
terras do Saco do Correia (44\$444) e a fazenda Saco do Moura, a cujas terras deu as seguintes divisas: confinam ao sul com a fazenda da Pintada, ao norte com a fazenda Catinga, ao poente com o Saco do Correia, ao moroeste coma Serrinha e a leste com a fazenda da Catinga (300\$000).

— José da Silva e Oliveira foi inventariado em 1840 e sua mulher Anna Ritta de Jesus, filha de Antonio Ferreira, em 1831. Terras, em um e outro inventario, as da fazenda Patos, herança do pae da inventariada. Filhos — Bernardo da Silva e Oliveira, o mais velho, com 20 annos em 1831, Ludovina, Maria do Espirito Santo, João Ferreira da Silva, Leonor da Silva e Oliveira, que se casou com Antonio Gonçalves Pereira, Manoel, Carlota e Clementina, esta, a mais moça com 3 annos.

— Em 1857 Anna Moreira fez partilha amigavel com seus filhos e genros por morte de seu marido alferes José da Silva Carneiro. Deu a inventario, entre outros bens, moveis, semoventes e bemfeitorias, as terras das fazendas Catinga, conforme as escripturas (12:000\$000), Tanque (6:000\$000), e a casa na parte pertencente ao casal, Poções, (5:000\$000), Cedro (2:000\$000), Bom Sucesso (560\$000), Bebedor, unidas ás do Cedro (324\$000), Lagedo (100\$000), Genipapo (300\$000), sortes de terras nas fazendas do Sitio, unidas  ás da fazenda Catinga (100\$000), da Serrinha, compra e herança, com posse no logar Páo Ferro (500\$000), Dois Irmãos (30\$000), Terra Nova (400\$000), Bôa Vista, logar Caldeirão do Tigre (500\$000), Massaranduba, (526\$000), Serra Grande (450\$000), S. Rosa (156\$000) Tiririca (250\$000), Algodões (150\$000), Maxixe (150\$000); a fazenda Mucambo, em terras da fazenda Catinga, uma casa no arraial da Serrinha, inclusive dois quartos da casa visinha comprados a João Manoel da Silva (500\$000), duas posses nos terrenos do patrimonio da padroeira da freguezia de Serrinha, uma que foi de Tristão Gomes da Silva e outra do finado Manoel José Moreira (50\$000).

Nas terras da Catinga foram aquinhoados a viuva, metade, para os lados do Mucambo, e os herdeiros Joaquim Moreira, capitão José Carneiro, María da Representação e Bernarda.

A fazenda Tanque coube, metade, á viuva, e ao herdeiro Ricardo, no logar Lagôa do Boi,
onde já está situado. Poções coube á viuva, e a Ricardo, Tristão e Francisco Simplício. As terras de Algodões e Genipapo ficaram para Fraucisco Simplício, as de Bebedouro e Cedro para Antonio Alves Carneiro, as de Bom Successo, Serra Grande e Dois Irmãos a Tristão, as de Lagedo, Caldeirão do Tigre e S. Rosa ao capitão José Carneiro, as de Serrinha à viuva, a Maria da Representação e a Joaquim Carneiro, as da Tiririca a Joaquim Carneiro, as do Sitio à viuva.

— Em 27 de Maio de 1876 os filhos de Maria Francisca da Purificação, viuva do capitão
José Carneiro da Silva, partilharam entre si os bens deixados pela mesma, cntre elles a fazenda Porteira, (10:000\$000), metde das terras da fazenda Serra Vermelha (4:500\$000), as terras das fazendas S. Rosa(300\$000) e Sitio (300\$000), estas compradas a João Cardoso Ribeiro, a fazenda Mucambo, em terras da fazenda Sitio, inclusive as bemfeitorias compradas a Antonio Carneiro da Silva, fallecido, em terras tambem da fazenda Sitio, comprehendendo todo o terreno desde o caminho velho que vinha do Mucambo velho para a casa do dito Antonio Carneiro riacho acima até topar as roças de José Affonso Ribeiro.

—O inventario do capitão José Carneiro da Silva foi feito em 1860, A viuva, Maria Francisca da Purificação, teve a fazenda Porteiras (14:000\$000) e metade da fazenda Serra Vermelha (4:500\$000).


— Por escriptura publica de 10 de Junho de 1717, na cidade do Salvador, Bahia de Todos os Santos, e pousadas de D. Izabel Maria Guedes de Britto, viuva do coronel Antonio da Silva Pimentel, esta, pelo cartorio do Tabellião Francisco Alves Camera, vendeu ao sargento-mor Thomé Pereira Pinto, morador no termo da villa de Cachoeira, «umas terras e citios: de criar gado, citas no certão do Itapicurú, que as houve de herança de seu pae o mestre de campo Antonio Guedes de Britto, entre os quaes são tres misticos, á saber, o citio Curral Novo, aonde está de arrendamento José Pereira Mascarenhas, que contronta e partirá pela parte do sul e nascente como citio do Rosario, terras do comprador, e pela parte do norte, correndo pelo rio do Itapicurú acima athé a partir com o citio das Queimadas, terras que hoje são de João Pirés Coelho, e pela parte do poente entre o Curral Novo e o Sitio das Queimadas cortará rumo direito a buscar o descançadouro do rio do peixe, citio em que está de renda Antonio Barbosa de Mendonça, terras ca mesma vendedora, digo que cortará rumo direito ao cítio do Curral Novo, ao citio das Queimadas, meya legua acima do descançadouro do rio do peixe e dahi seguirá pelo rio do peixe abaixo até topar com o mesmo citio em que está de renda Antonio Barbosa de Mendonça, terras da vendedora, e correndo pelo rio do peixe abaixo até topar o citio da Cruz, terras que foram de Manoel Garcia Pimentel, que são hoje do comprador, e dahi cortará pela outra parte do rio do Peixe da parte do sul meya legua ao Certão acompanhando o dito rio, athé acima do descançadouro do rio do peixe meya legua, os quaes citios ribeira acima confrontados e demarcados, da, mesma forma que ella vendedora possue e de antes seus antecessores, vendem ao dito comprador sargento-mór Thomé Pereira Pinto por 1:000\$000, que são dois mil e quinhentos cruzados.» Testemunhas - Valerio de Moura e João Rodrigues Garcia, criados da vendedora.

— Por escriptura publica de 20 de Fevereiro de 1797, passada na cidade do Salvador, Bahia de Todos os Santos, e cartorio do tabellião João Damasio José, d. Francisca Joanna Josepha da Camara, viuva do illustrissimo Manoel de Saldanha, moradora em Lysbôa, por seu procurador, capitão Manoel Pinto da Cunha vendeu a Manoel Telles de Almeida, por seu procurador Antonio José de Mello, uma fazenda de criar gados cita no certão dos Tocós, Termo da Villa de itapicurú de Cima, denominada Carahybas e seu retiro descançadouro que parte e confronta pela parte do sul começando o rumo do logar chamado Malhada do Mandacarú, correndo pelo nascente buscando o morte athé a barra chamada do Rio Verde, e dahi fica partindo com a escriptura da fazenda do Mucambo, tudo para a parte do nascente, pela banda do norte partindo ao meyo a terra que fica entre a serra do Mucambo e logar chamado Tanquinho, seguindo ahi o rumo direito da mesma terra Mucambo e pegando o rumo do nascente ao poente thé topar com as terras que foram de d. Izabel chamada o Curral Novo, e na mesma conformidade com terra de S. Antonio, e Rio d'Agua, conforme os seus titulos, para a banda do sul partindo o rumo do dito logar chamado Malhada do Mandacarú a buscar a estrada das boiadas correndo o mesmo rumo direito, partindo ao meio coma fazenda do Rio do Peixe de Sima de Antonio Manoel de Carvalho, cuja fazenda assim confrontada e declarada, vende por 400\$000. Testemunhas João de Deus Telles de Menezes e José da Costa de Abreu e João Francisco da Silva, Manoel Telles de Almeida, por escripto particular datado da Fazenda Descançadoro 23 de Abril de 1797, cedeu metade dessa fazenda Carahiba e seu Retiro a seu primo João Carvalho Gomes. Testemunhas — José Pinto da Fonseca, que assiguou à rogo do cedente, por ser analphabeto, Miguel Telles de Almeida e Antonio José de Mello.

— Por escripto particular datado de Serrinha 19 de Fevereiro de 1825 Maria Rita Pinheiro vendeu a Luiz Ferreira de Araujo, Theodozio Vaz de Aguiar e José Garcia de Araujo por 200\$000 uma sorte de terras na fazenda do Rosario, que houve por herança de seus paes o capitão Manoel de Affonseca Pinheiro e Anna Maria da Silva. Pela vendedora assignou seu irmão José Pinheiro de Affonseca. Testemunhas — Vicente Ferreira da Silva e José Pereira Pinto.

— Por escripto particular datado de Capella da Serrinha 22 de Outubro de 1836 Antonio de Mattos Paim e sua mulher Anna Maria da Silva venderam a Jacintho Garcia de Araujo por 200\$000 um quinhão de terras sitas na fazenda Rosario,
termo da villa e comarca de Itapicuru de Cima, que houveram por titulo de compra de seu irmão e cunhado José Pinheiro de Affonseca, em commum com os demais herdeiros. Pela vendedora assignou Vicente Ferreira da Silva, Testemunhas - Affonso da Silva Cardoso, José Pinheiro de Affonseca e Francisco Joaquim de Araujo.




— Por escripto particular datado de Queimadas 22 de Fevereiro de 1855 José Joaquim de Almeida e sua mulher Ignacia Maria de Jesus, por seu procurador Egidio José de Araujo, venderam a José Maria de Lacerda Ribeiro uma sorte de terras no logar denominado Rosario, com suas bemfeitorias que houveram por compra a José Pereita Pinto e sua mulher d. Antonia da Silva Pinheiro.

- Por escripto particular de Rosario 6 de Fevereiro de 1860, Antonio Joaquim de S. Anna e sua mulher Maria Victoria de Jesus venderam por 50\$000 ao seu afilhado José Maria de Lacerda Ribeiro  uma porção de terras na fazenda Rosario, que houveram por compra feita a José Pereira Pinto e sua mulher Antonia Pereira da Silva. Pela vendedora assignou seu filho  João Ferreira  de S. Anna. Testemunha Manoel José da  Cruz.

- José Marcellino Rodrigues, casado com Maria Francisca, filha de Anna Francisca e José Manoel da Motta, Tambuatá, teve cinco filhos, entre homens e mulheres, dos quaes são vivos Francisco Manoel da Motta e outro.

- Por escriptura publica de 22 de Setembro de 1814 «neste arraial da Serrinha», freguezia e termo da villa de S. Joãode Agua-Fria, tabellião Manoel Carlos de Saraiva Belfort, testemunhas Manoel José Moreira e Francisco Manoel da Cunha a Illustrissima e Excellentissima Condessa da Ponte, D. Maria Constança de Saldanha Oliveira e Souza, por seu procurador Antonio Manoel da Silva, morador neste Termo, «como tutora e administradora por carta regia das pessoas e bens de seus filhos menores, bem  assim senhora e possuidora de varias fazendas citas no logar dos Tocós e Pindá e por  uma convenção que seu antecessor o mestre de campo Antonio Guedes de Britto havia feito com João Peixoto Viegas, em que contrataram cortar um rumo do logar Jacuhype Velho à serra do Irará? e que todas as terras que ficarem do rumo dito para o norte ficarão pertencendo á casa delle dito mestre de campo e hoje de sua
constituinte e que as que ficarem para a parte do sul ficarão pertencendo ao dito Viegas, cujas terras tinha havido por titulo de herança de seu fallecido marido o excellentissimo Conde da Ponte, este de seu pae o excellentissimo Ma
noel de Saldauha da Gama Guedes de Britto, e este de sua mulher em primeiras tupcias d. Joanna da Silva Guedes de Britto e esta de seu pae o coronel Antonio da Silva Pimentel e de sua mãe d. Izabel Maria Guedes de Britto, esta
de seu pae o mestre de campo Antonio Guedes de Britto, este juntamente com seu pae Antonio de Britto Correia por doação real e remuneração dos relevantes serviços, que prestaram ao Estado como conquistadores dos certões desta Capitania e circumvisinhanças, e porque algumas das fazendas a ella outhorgante pertencentes foram vendidas pelos predecessores do dito João Peixoto Viegas, prevendo que ficariam da parte sul da linha divisoria, que se devia correr para a parte do sul e uma dellas é a fazenda S. Bartholomeu, de que se acha de posse Antonio da Silva
Carneiro, porisso e por este não querer comprar as ditas terras, e querer arrastar as consequencias judiciarias e a incerteza de seu resultado está informada de que ella se acha ao norte da dita linha divisoria estava justo e contractado amigavelmente a vender, como com effeito vendia de hoje para todo sempre ao outhorgante comprador José da Silva Carneiro todo direito, acção e pretenção que a casa da Illustrissima sua constituinte tinha e podia a vir ter ás terras da dita fazenda de S. Bartholomeu, que se divide pela forma seguinte: a saber, partindo ao meio com as terras das fazendas Possos, Cedro e Barra, e onde fizer meio com esta fazenda, e procurará o morro da malhuda da Lagôa e deste irá até encontrar o meio que foi a fazenda do Viado com a fazenda de São Bartholomeu e deste logar procurará a serra do Taboleiro Alto até o Boqueirão e deste a procurar um morrinho alto? que existe por detraz do tanque novo e dahi sahirá até encontrar o meio que foi a fazenda dos Vermelhos. E voltando ao norte, irá até topar com terras da fazenda do Bom Successo ja vendida pela illustrissima casa da outhorgante tendo de comprimento uma legua para a parte da fazenda das pedras, cuja terra, assim dividida e demarcada com todos os seus matos, pastos, logradouros, entradas e sahidas e com todos os mais commodos que na dita terra se acharem, acção e pretenção que nella podia ter venda, como com effeito vendido tinha, de hoje para todo o sempre, ao outhorgante comprador José da Silva Carneiro pelo preço e quantia de quatrocentos mil reis» em prestações annuaes de 50\$000 e juros. Antonio Carneiro da Silva tambem assignou a escriptura «em signal de que cedeu da compra e não quer comprar». Escrip. pub. por traslado dos livros de notas de fls. 114 a fls: 117, tirada pelo tabellião da villa de N. S. da Purificação dos Campos, Joaquim José da Costa, em 15 de Julho de 1845.

— Em 1761 o tenente-coronel de engenheiros Manoel Cardoso de Saldasha, o capitão de infanteria Francisco da Cunha Araujo e o coronel Leobino Mariz desempenharam-se da commissão relativa aos salitres de Montes Altos, O
capitão Francisco da Cunha Araujo era do municipio de Cachoeira. Será parente dos Cunha Araujo, do Coité?

%—O morador do sitio Massaranduba, em 1716, não era o capitão Antonio Alvares Filgueiras, como por engano se diz na pagina \pageref{filgueiras}, mas o capitão João Alvares Filgueiras. %corrigido

— Jacobina foi criada villa por carta regia de 5 de Agosto de 1720. Iustallada aos 24 de Junho de 1722 no sitio Sahy, missão de N. S. das Neves, e freguezia de S. Antonio de Jacobina, pelo coronel Pedro Barbosa Leal, fidalgo da Casa de S. Magestade e cavalheiro professo da ordem de Christo, foi, em virtude de reclamação do coronel Garcia d' Avila Pereira ao reo foi transferida para o sítio e arraial da miissão do Bom Jesus, onde se instalou em 5 de Agosto de 1724 e aínda hoje se acha. A transferencia foi feita por ordem do governador geral Marcos Fernandes Cezar de Menezes, em consequencias de ordens da Corôa, de 26 de Abril de 1724, ao dezembargador dr. Ouvidor geral da comarca Pedro Gonçalves Cordeiro Pereira.

— Diz o padre Cupertino que Francisca, filha do tenente Manoel de Jesus, neta de Manoel de Jesus e bisneta do capitão Apollinario, foi casada com João Caetano Gonçalves de Castro e não teve filhos.

— O capitão Manoel de Affonseca Pinheiro, genro de Miguel Affonso Ribeiro, teve do seu casal sete filhos, que são os mesmos cujos nomes declinamos á pag. 33 e mais Maria Ritta, solteira, que o typographo eliminou do numero de seus filhos, engolindo palavras e repetindo outras. Repetidas, e por isso mesmo devem ser eliminadas, são as palavras — \textit{mais netos lhe deu}, que se seguem ás palavras - \textit{dos padres Loreto}, bem como as cinco linhas finaes da mesma pagina 33 e as cinco primeiras linhas da pagina 34. Depois da palavra — Loreto deve se ler, na linha 25 da pagina 33, \textit{e Urbano}. O mais, como está, feitas as eliminações recommendadas, de modo que se leia — e pae dos padres Loreto e Urbano (José Alves Martins do Loreto e Urbano Cecilio Martins), o qual, tendo enviuvado, etc., etc., etc.

%—A mulher de Antonio Ferreira da Silva Pinheiro, pagina \pageref{vfgoes}, chamava-se Virginia Ferreira Gomes, filha de José Gomes, e não V. F. de Góes, filha de José de Góes. %não achei esse erro

— Jesuina, casada com João Paes Cardoso, pag. \pageref{jesuina}, morreu em 1856 e deixou tres filhos: José, Manoel e Antonio.

— O marido de Antonia, filha de Antonio de Mattos Paim, chamava-se Manoel, pag. \pageref{ampaim}. Tiveram os seguintes filhos:  Maria Joanna de Mattos Paim, Leonilla de Mattos Paim, Maria da Paixão Pinheiro, casada com Tirço da Silva Pinheiro, Anna de Mattos Paim, Cecilia de Mattos Paim e Theresa de Mattos Paim.

—Maria Magalena de Pozia e Jovina, pag. \pageref{mmagdalena}, não são filhas de Virginia Ferreira Gomes, mas irmãs. As tres são filhas de José Gomes, morto antes de 1867, e de Paulina de Souza Gomes, sua mulher, inventariada em 1867. Antonio Ferreira da Silva Pinheiro, pois, é cunhado e não pae de Maria Magdalena e de Jovina.

— João Paes Cardoso, pagina \pageref{jesuina}, foi casado em primeiras nupcias com Jesuina e em segundas nupcias com Maria Militina de Jesus, filha de Joaquim Affonso e de Maria Francisca de Jesus.

%— A' pagina \pageref{lenha}, linha 12, leia-se Unha do Gato, e não, como está, lenha do Gato. %corrigido

— Por escriptura publica de 5 de Maio de 1778, passada em Jacobina, Maria Magdalena Pereira (ou Ferreira?) moradora na freguezia velha da villa de Santo Antonio da Jacobina, constituíu seus procuradores em Agua Fria a Apollinario Ferreira (ou Pereira?) Fructuoso de Oliveira e Miguel Affonso.

— Por escriptura publica de 14 de Novembro É 1821 Antonio de Mattos Paim e sua mulher Anna Maria da silva, segundas nupcias, deram bens em Jacobina, em garantia da legitima de Anna Felicia, filha do primeiro e enteada da segunda no valor de 26\$727. Foi fiadora Sua mae d. Anna Innocencia da Silva, viuva.

— Aos filhos de Vicente Ferreira Ramos e sua mulher Maria Francisca de Jesus, enumerados á pagina \pageref{vframos}, é preciso accrescentar Anna Maria de Jesus casada com José Maria de Oliveira; Francisca Maria de Jesus, casada com José Gonçalves de Oliveira e Theodora Maria de Jesus casada com Zacharias Ferreira da Silva; e aos Filhos de Marianna Maria de Jesus e seu marido Joaquim Pinheiro de Carvalho, netos de Vicente Ferreira Ramos e Maria Francisca de Jesus, é preciso ajuntar Anna Pinheiro, casada com Ignacio Carneiro e Izabel.

— José Pinheiro Alves de Souza, página \pageref{jpalves}, foi inventariado em 1860. Era viuvo de Ignacia de cujo consorcio teve os seguintes filhos: Antonio Alves Pinheiro, Pedro Alves Pinheiro, Maria Ramos e Anna Christina, esta então falecida, casada que foi com José Maria Ferreira da Motta, de quem deixou um filho, Virginio.

— Observe-se à pagina \pageref{fgpereira}, ultimas linhas o seguinte:
Em 1842 foi inventariado Francisco Gonçalves Pereira Junior, que fôra casado com Maria Ritta de Jesus, deixando os filhos seguintes: Joanna, Maria, Rosa e Vicencia, e terras nas fazendas Fundo, Emburana e Campinas. Figura tambem com o nome de Francisco Gonçalves Somente e morreu em Setembro de 1842. Joanna, filha mais velha, tinha seis annos e Vicencia 6 mezes.

— Em 1850 Francisca Maria do Espirito-Santo deu a inventario os bens de seu marido Manoel da Costa.
Terras — as do Sitio Sicupira (400\$000),
Herdeiros — Antonio Manoel da Costa, José Caetano, José da Costa, Manoel Francisco, Maria dos Santos, José Antonio da Costa, João da Costa, Lino da Costa, Francisco da: Costa, Anna Maria e José Manoel da Costa, fallecido e representado por seus filhos Joanna casada com Manoel José dos Sautos, Maria, Patrício e Francisca,

— Entre os filhos de Antonio Ferreira de Oliveira e netos de José Ferreira de Oliveira, pagina \pageref{jfoliveira}, é preciso incluir Simão Ferreira de Oliveira, omittido pelo typographo, casado que foi com Maria Lina, filha de Matheus Carneiro, Mirante, Lino chamava-se Tertulino e não Tertuliano.

— Quem morreu antes de 1846 foi Maria Vicencia e não seu marido Joaquim Pereira Gonçalves, pag. \pageref{jpgoncalves}. Os filhos do casal foram Maria, casada com João Paulo de Araujo, Anna, casada com José Ferreira de Oliveira Gomes, Maria Francisca, Maria Clementina, Antonia Bernarda, Joaquim, Jesuina e Ludovina.

— Manoel Ferreira de Oliveira e sua mulher Maria da Penha, pag. \pageref{mpenha}, ainda tiveram uma filha, Maria, que se casou com Manoel Braz.

— A' linha 28 da pagina \pageref{maleuteria} leia-se: Maria Eleuteria casou-se com Jose Gonçalves Pereira, do Exú, Custodia casou-se com Francisco Brazileiro; e o resto como está escripto.

— Nove foram os filhos de José Ferreira de Carvalho, pag. \pageref{jfcarvalho}. Entre elles Ritta Constantina, casada com Virginio Ferreira de Oliveira, ahi não incluida.

% O marido de Thereza, pag.\pageref{elisio} , linha antepenultima, não era Joaquim Gonçalves Pereira, porem, sim, Joaquim Elísio Pereira. %corrigido

— São tetranetos de Bernardo da Silva, por via de Antonia Maria, e sua filha Bernarda, Deca, filho de Manoel Gonçalves Pereira e José Joaquim de Oliveira, filho de Antonio Joaquim. Dil-o o padre Cupertino.


%— A Antonia Maria da Silva, pag. \pageref{anmsilva}, acrescente-se — e Anna Maria da Silva. %corrigido

%— Foi a fazenda S. Rita, e não a fazenda Agua Fria, que foi fundada pelo marido de uma filha de Antonio Carneiro da Silva, pag. \pageref{starita}. %corrigido

— Uma das testemunhas da posse dos sitios Saquinho e Passos, dada judicialmente a Fructuoso de Oliveira Maia, em 1759, foi Bernardo da Silvas Vide pag. 93. Será o fundador da Sereinha ou um seu filho do mesmo nome? O professor Martins diz que Bernardo teve um filho do seu nome. Mas parece que Bernardo em 1759 era vivo.

— Em 5 de Setembro de 1925 (vide a collecção do «O Serrinhense») habilitaram-se para casamento João Onofre de Araujo e Paulina Maria de Jesus; elle viuvo, com 50 annos de idade, natural e residente no municipio de Serrinha, filho legitimo de Antonio Joaquim de Araujo e Virginia Maria de Jesus, ambos fallecidos; ella solteira, com 36 annos de idade, filha legitima de Guilhermino de Oliveira, já fallecido, e
Maria de Oliveira, residente na fazenda Queimada do Riachão do Jacuhype. Vide pag. \pageref{ma18} deste livro.

%— Tem o sobrenome de Oliveira, e não de Moreira, o coronel Aristides Cedraz. %corrigido

— Em 1844 foi inventariado Antonio Goncalves de Oliveira, morto em 4 de Dezembro de 1844. Foi inventariante a sua viuva Maria Ferreira. Herdeiros do primeiro casal: José Francisco de Oliveira, Manoel Francisco, Innocencio Gonçalves de Oliveira, Anna, João Manoel de Araujo, José Maria de Oliveira, Antonio Gonçalves, Victoriano de Oliveira, maiores, Maria, 16 annos, Eduarda, 14 annos menores, 2 mais Senhorinha, fallecida e representada por seus filhos Maria, Luiz, Joanna, Marianna e Joaquim, tambem fallecido e representado por seus filhos José e Innocencio. herdeiros do segundo casal não existem. Terras — as da fazenda Matto Grosso, perto da Chapada, Escrivão — Antonio Quintilhano Carneiro da Silva. Avaliadores — João Lopes Guimarães e José Alevandre de Araujo. A viuva morava na capella de N. S. da Conceição do Coité, bem como os herdeiros Innocencio José de Oliveira, José Maria de Oliveira, Antonio Manoel de Araujo, João Manoel de Araujo, Victoriano José de Oliveira e Marianna Francisca de Jesus, que por isso constituíram procurador. Nota-se alguma differença nos nomes dos herdeiros que passaram procuração e dos que constam do titulo de herdeiros.


— Em 1867 José Lopes da Silva inventariou sua mulher Anna Maria de Jesus. Herdeiros — seus filhos José, Joaquim TLeodorio, Manoel Lopes da Silva, Salustiano Lopes da Silva e Satyro. José, o mais velho, com 16 annos, e Satyro, o mais moço, com 10. Avaliadores — Tenente-coronel Joaquim Carneiro de Campos e Joaquim Quintiliino de Oliveira. Terras: as da fazenda Campo Limpo, em terras da fazenda Serra Grande (200\$000), e as da fazenda Pedra (80\$000).

— Em 1872 Joaquim Moreira da Silva Carneiro, Ricardo Carneiro da Silva Moreira, d. Maria Moreira da Representação Carneiro, d. Bernarda Verissima Carneiro, Francisco Simplício da Silva Carneiro, Antonio Alves Carneiro, filhos e genros da fallecida Anna Moreira da Silva Carneiro, fizeram partilha amigavel dos bens desta (27 de Abril de 1872.) Escrivão: Elpídio Carneiro da Silva Ribeiro. Terras: as da fazenda Mucambo, metade da outr'ora fazenda Catinga (6:000\$000), as da fazenda Sitio, uma parte ainda nos terrenos do Mucambo (120\$000), as de Serrinha (374\$000), as da fazenda Tanque (3:000\$000), as da fazenda Poções (3:100\$000), as da fazenda Cajueiro (400\$000).

— 1861. Inventariante José Baptista, Inventariada sua mulher Florencia de Jesus, morta em Agosto de 1861. Terras: 4\$000 de terras na fazenda Lamarão, herdadas de Maria de Assumpção. Nove herdeiros. A herdeira Anna Florencia casou-se com Lourenço José Baptista.

— 1875. Inventariante Carolina Marta do Espirito-Santo. Inventariado—seu marido Antonio Manoel Percira. Terras: as da fazenda Regallo. Herdeira unica: Maria Carolina do Espirito Santo, que se casou com Joaquim Cardoso Ribeiro, filho do capitão Antonio Cardoso Ribeiro, família dos Carneiros. Antonio Manoel Pereira era irmão de João José Pereira.

— 1873. Inventariante Joaquim Quintino de Oliveira. Inventariada: sua mulher Anna Ritta Carneiro de Oliveira. Não constam os nomes dos herdeiros, todos menores.

— 1854. Inventariado Joaquim Francisco Leal. Inventariante sua mulher Maria Eleuteria de Jesus. Filhos: Manoel Francisco da Rocha, Marianna casada com Manoel Thomaz de Sant'Anna, Maria Francisca de Jesus casada com José da Rocha Medrado, Antonia Maria dos Prazeres, viuva, Theodora, viuva, Izabel casada com Manoel Bernardo de Araujo, Appolinaria casada com Seraphim José de Araujo, Josepha casada com Januario de Mattos, Luiza casada com Caetano José de Almeida, José Joaquim S. Paio, Maria Francisca, fallecida e representada por seus filhos Izidro, Maria e Antonio (3). Terras: as da fazendas Côxo, no Coité (650\$). Avaliadores: José Francisco de Araujo e João Manoel de Araujo. Foi comtemplado na partilha, por cabeça de sua mãe, Fortunato José de Araujo, que não consta do titulo de herdeiros.

— Em 1883 foi inventariado Manoel Joaquim de Araujo Oliveira, morto em 11 de Agosto de 1883. Inventariante: sua viuva Maria de Lima Carneiro e Oliveira. Herdeira unica: Anna. Terras: as da Tabúa. (200\$000).

— 1859, Inventariante Manoel José da Silva. Inventariada: sua mulher Maria faustina de Lima. Herdeiros: Anna Rosa, fallecida e representada por seu pae, Temotheo, Henrique, Joanna e Marcellina. Terras: as do Alto das Candeias (120\$000), em commum com o Genipapo.

— 1853. Inventariante Antonio Manoel da Motta. Inventariada Maria Francisca de Jesus. Herdeiros: Francisco, Senhorinha e Maria (2 annos). Terras: as da Formiga (100\$000).

— 1886. Inventariante Maria Francisca de Jesus. Inventariado seu marido Antonio Manoel da Motta. Herdeiros do 1º casal: Francisco Militão da Motta, Senhorinha Virgem de Portugal casada que foi com Antonio Joaquim Ramos de Almeida, representada por seus filhos Maria, Francellina, João, Manoel e Adolfo (5). Herdeiros do 2º casal: Maria Francisca de Jesus, Anna, Maria de Jesus, Joaquim e Antonio (4). Terras: as da fazenda Retiro (200\$000). Francisco Militão casou-se com Josepha Raymunda de Jesus.

— 1842. Inventariante Izabel Perpetua de Jesus. Inventariado: seu marido Francisco Borges de Oliveira. Herdeira unica: sua filha Maria Amelia de Jesus. Terras: as de S. Rosa (30\$).

— Em 1874 Joaquim Affonso da Silva deu à inventario os bens de sua fallecida mãe Maria Francisca de Jesus. Herdeiros: Joaquim Affonso da Silva, José Gregorio da Silva, Maurícia Apollinaria, Braz Ferreira de Araujo - por cabeça de sua mulher Luiza Alberta, Torquato Lino da Silva (maiores), Maria Militina de Jesus, casada que foi com João Paz Cardoso, fallecida e representada por seus filhos Catharina, Maria, Anna, Joanna e Maria da Cruz. Terras: as de Serrinha (62\$700) e as de Serra Grande (30\$).

—1858. Inventariante: Joaquim Ferreira Santhiago. Ivventariada: sua mulher Maria de Jesus. Filhos: Silvestre e José. Terras: O sitio Lage do Mocó (50\$000) e o sitio Tanque Novo (300\$000).

— 1860. Inventariante:  Pedro  José da Silva. Inventariada: Maria Fiminiana de jesus. Terras: as do Saquinho. Herdeiros: Estanisláo, Maria, Anna, Tertulino, Margarida e Evaristo, todos filhos.

— 1855. Inventariante: Ignez Maria de Jesus. Inventariado: O seu marido Manoel Francisco Barbosa. Terras: As da fazenda Genipapo (300\$000). Filhos: Antonio, Anna, Rufina, Carlota, Marcos.

— 1864. Inventariante: Anna Josepha de Jesus. Inventariado: O seu marido Manoel Ferreira da Silva. Filhos: Joanna, Fortunata, Bazilio, Carlota, Zeferina, Maria Victoria e Margarida. Terras: As do Curralinho (50\$000) e as do Subaé misticas com a fazenda Curralinho (80\$000).

— 1869. Inventariante: Honorata Maria de Jesus. Inventariado: seu marido Matheus Cardoso Valverde. Filhos: José Melchiades, Antonio Ferreira da Motta (por sua mulher Maria  Magdalena), Andre  Avelino (por sua mulher Thomazia)e José Felix. Terras: As da fazenda Rosario (200\$000), as do Boqueirão (50\$000), as da Massaranduba (30\$000) e as do Brejo na fazenda da Serra Grande (50\$000).

— 1869. Inventariante: Pedro Alves Silva. Inventariada: sua mulher Ignacia Maria de Jesus. Filhos: José, Luiz, Justino, Anna e Manoel, estes dois ultimos mortos depois de sua mãe. Terras: as da fazenda Larangeiras (150\$), Rosario (50\$000), Riacho da Manga, nos suburbios deste arraial de Serrinha (50\$000). Pedro Alves da Silva é tio de Manoel, Maria, Anna, Antonia e Constança, filhos de Vicente Ferreira de Araujo e sua mulher Maria Alexandrina da Silva, esta inventariada em 1866, deixando terras no Tambuatá (50\$000), Maria (Maria Margarida de Jesus), sua sobrinha, casou-se com Manoel Pinheiro de Carvalho.

—Em 1866 foi inventariado Pedro José da Silva. Inventariante: André José de Medeiros. Herdeiros: Estanisláo, Isabel, Maria, Anna, Tertulina, Margarida e Evaristo, todos filhos. Terras: as da fazenda Saco do Correia.

— Por escriptura publica de 9 de Novembro de 1873 Maria Francisca da Purificação doou a fazenda Porteiras a seus filhos tenente-coronel Miguel Carneiro da Silva Ribeiro, Joaquim Carneiro Ribeiro, José Carneiro da Silva, Antonio Martins Ferreira () por cabeça de sua mulher Anna Carneiro da Silva), tenente Manoel Cardoso Ribeiro (por cabeça de sua mulher Maria Rosa Carneiro), capitão Tertuliano Carneiro da Silva Ribeiro, capitão Antonio Carneiro da Silva Ribeiro e Elpídio Carneiro da Silva Ribeiro,

— João Ferreira da Silva, Badé, e Manoel Ferreira da Silva, Badé, são irmãos de Custodio Ferreira da Silva e Maria da Silva Oliveira, casada com Agostinho José de Oliveira, paes do coronel Francisco Augusto de Oliveira, de Monte Alegre, vivo e maior de 60 anos, são filhos de Rosaria, informa-me o mesmo coronel Francisco Augusto, que aliás ignora o nome do marido de Rosaria.

%— Chamava-se Josephina, pagina \pageref{josephina}. ultima linha, a mulher de José Ferreira da Siva. %corrigido

— Manoel José Pinto, portuguez, casou-se com Bernarda Archanja Moreira em 4 de Setembro de 1827. A sua filha Antonia Clementina Moreira casou-se com o capitão José Joaquim-de Araujo em 18 de Setembro de 1856 e desse consorcio teve os seguintes filhos: João, nati-morto, em 12 de Maio de 1857; padre José de Cupertino e Araujo Lima, vigario de S. Anna do Catú, ex-deputado provincial, nascido em 12 de Setembro de 1858; Cecilia, Dadate, nascida eim 20 de Novêmbro de 1859, solteira; Maria, nascida em 17 e morta em 23 de Setembro de 1861; Maria Herminia, viuva do capitão Symphronio Cardoso Ribeiro, nascida em 1 de Setembro de 1862; Reginaldo Cyrillo de Araujo, fallecido em plena adolescencia, nascido em 28 de Janeiro de 1865; Laudelina Candida, Sinhá Dona, nascida em 22 de Abril de 1866, solteira; Antonio, nascido em Março de 1868 e fallecido em 5 de Junho do mesmo anno; Bacharel Antonio José de Araujo, nascido em 8 de Maio de 1869, actualmente Juiz de Direito da comarca de Jacobina, casado em 8 de Dezembro de 1892 com d. Guilhermina de Castro Araujo, filha do major Alberto Moreira Castro e d. Maria Sophia Gomes de Castro, da cidade de Lençóes,comarca de Lavras Diamantinas. Joaquim José de Araujo, nascido em 24 de Agosto de 1870 e fallecido em 11 de Janeiro de 1886, victima de Febre amarella, que então grassava em Serrinha, onde se achava em goso de ferias collegiaes. 


— Por escriptura publica de 22 de Março de 1755 o sargento-mor João dos Santos Reis, como procurador do coronel João Peixoto Viegas e sua mulher d. Ritta Josepha Brandão, vendeu ao coronel Manoel de Figueiredo Mascarenhas «dois citios de terras de criar gados chamados São Leandro e outro Ambuzeiro, citos á beira do rio Jacoipe, o qual citio de São Leandro: se começará e principiará a medir do citio da Vargem Grande onde acabar a terra do mesmo citio que rematou o tenente coronel Lourenço Correya Lysbôa e dahi correrá pelo mesmo rio Jacoipe acima até encher e completar o numero de cinco leguas de terra de comprido para o certão e terra a dentro buscando o nascente e sul terá de largura tres leguas em todo o comprimento das cinco leguas; e da mesma Sorte o citio Ambuzeiro principiará e se começará a medir para à parte da aldeia de São João do Jacoipe na varge chamada do Piripiri, correndo pelo mesmo rio Jacoipe acima até encher o numero de outras cinco leguas de terra de comprido e para o certão e terra  a dentro buscando o sul terá de largura tres leguas em todo o comprimento das cinco leguas e para a parte do poente e norte partem os ditos dois citios desta venda pelo mesmo rio Jacoipe com  terras de Manoel de Saldanha, os quaes dois citios de terra São Leandro e Ambuzeiro assim confrontados e demarcados como todos os mattos, varges, ribeiros, lagôas, catingas, vallados, campos e brejos que houver e se achar dentro das terras dos ditos dois citios, o coronel João Peixoto Viegas e sua mulher d. Ritta Viegas Brandão, vendem ao comprador o coronel Manoel de Figueiredo Mascarenhas por 800\$000». A escriptura foi passada na villa e Minas de Jacobina e o coronel Manoel de Figueiredo Mascarenhas era morador no citio Jaboticabas, distante tres leguas da cidade de Jacobina, então villa.




